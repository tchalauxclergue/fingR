% Options for packages loaded elsewhere
\PassOptionsToPackage{unicode}{hyperref}
\PassOptionsToPackage{hyphens}{url}
%
\documentclass[
]{article}
\usepackage{amsmath,amssymb}
\usepackage{iftex}
\ifPDFTeX
  \usepackage[T1]{fontenc}
  \usepackage[utf8]{inputenc}
  \usepackage{textcomp} % provide euro and other symbols
\else % if luatex or xetex
  \usepackage{unicode-math} % this also loads fontspec
  \defaultfontfeatures{Scale=MatchLowercase}
  \defaultfontfeatures[\rmfamily]{Ligatures=TeX,Scale=1}
\fi
\usepackage{lmodern}
\ifPDFTeX\else
  % xetex/luatex font selection
\fi
% Use upquote if available, for straight quotes in verbatim environments
\IfFileExists{upquote.sty}{\usepackage{upquote}}{}
\IfFileExists{microtype.sty}{% use microtype if available
  \usepackage[]{microtype}
  \UseMicrotypeSet[protrusion]{basicmath} % disable protrusion for tt fonts
}{}
\makeatletter
\@ifundefined{KOMAClassName}{% if non-KOMA class
  \IfFileExists{parskip.sty}{%
    \usepackage{parskip}
  }{% else
    \setlength{\parindent}{0pt}
    \setlength{\parskip}{6pt plus 2pt minus 1pt}}
}{% if KOMA class
  \KOMAoptions{parskip=half}}
\makeatother
\usepackage{xcolor}
\usepackage[margin=1in]{geometry}
\usepackage{color}
\usepackage{fancyvrb}
\newcommand{\VerbBar}{|}
\newcommand{\VERB}{\Verb[commandchars=\\\{\}]}
\DefineVerbatimEnvironment{Highlighting}{Verbatim}{commandchars=\\\{\}}
% Add ',fontsize=\small' for more characters per line
\usepackage{framed}
\definecolor{shadecolor}{RGB}{248,248,248}
\newenvironment{Shaded}{\begin{snugshade}}{\end{snugshade}}
\newcommand{\AlertTok}[1]{\textcolor[rgb]{0.94,0.16,0.16}{#1}}
\newcommand{\AnnotationTok}[1]{\textcolor[rgb]{0.56,0.35,0.01}{\textbf{\textit{#1}}}}
\newcommand{\AttributeTok}[1]{\textcolor[rgb]{0.13,0.29,0.53}{#1}}
\newcommand{\BaseNTok}[1]{\textcolor[rgb]{0.00,0.00,0.81}{#1}}
\newcommand{\BuiltInTok}[1]{#1}
\newcommand{\CharTok}[1]{\textcolor[rgb]{0.31,0.60,0.02}{#1}}
\newcommand{\CommentTok}[1]{\textcolor[rgb]{0.56,0.35,0.01}{\textit{#1}}}
\newcommand{\CommentVarTok}[1]{\textcolor[rgb]{0.56,0.35,0.01}{\textbf{\textit{#1}}}}
\newcommand{\ConstantTok}[1]{\textcolor[rgb]{0.56,0.35,0.01}{#1}}
\newcommand{\ControlFlowTok}[1]{\textcolor[rgb]{0.13,0.29,0.53}{\textbf{#1}}}
\newcommand{\DataTypeTok}[1]{\textcolor[rgb]{0.13,0.29,0.53}{#1}}
\newcommand{\DecValTok}[1]{\textcolor[rgb]{0.00,0.00,0.81}{#1}}
\newcommand{\DocumentationTok}[1]{\textcolor[rgb]{0.56,0.35,0.01}{\textbf{\textit{#1}}}}
\newcommand{\ErrorTok}[1]{\textcolor[rgb]{0.64,0.00,0.00}{\textbf{#1}}}
\newcommand{\ExtensionTok}[1]{#1}
\newcommand{\FloatTok}[1]{\textcolor[rgb]{0.00,0.00,0.81}{#1}}
\newcommand{\FunctionTok}[1]{\textcolor[rgb]{0.13,0.29,0.53}{\textbf{#1}}}
\newcommand{\ImportTok}[1]{#1}
\newcommand{\InformationTok}[1]{\textcolor[rgb]{0.56,0.35,0.01}{\textbf{\textit{#1}}}}
\newcommand{\KeywordTok}[1]{\textcolor[rgb]{0.13,0.29,0.53}{\textbf{#1}}}
\newcommand{\NormalTok}[1]{#1}
\newcommand{\OperatorTok}[1]{\textcolor[rgb]{0.81,0.36,0.00}{\textbf{#1}}}
\newcommand{\OtherTok}[1]{\textcolor[rgb]{0.56,0.35,0.01}{#1}}
\newcommand{\PreprocessorTok}[1]{\textcolor[rgb]{0.56,0.35,0.01}{\textit{#1}}}
\newcommand{\RegionMarkerTok}[1]{#1}
\newcommand{\SpecialCharTok}[1]{\textcolor[rgb]{0.81,0.36,0.00}{\textbf{#1}}}
\newcommand{\SpecialStringTok}[1]{\textcolor[rgb]{0.31,0.60,0.02}{#1}}
\newcommand{\StringTok}[1]{\textcolor[rgb]{0.31,0.60,0.02}{#1}}
\newcommand{\VariableTok}[1]{\textcolor[rgb]{0.00,0.00,0.00}{#1}}
\newcommand{\VerbatimStringTok}[1]{\textcolor[rgb]{0.31,0.60,0.02}{#1}}
\newcommand{\WarningTok}[1]{\textcolor[rgb]{0.56,0.35,0.01}{\textbf{\textit{#1}}}}
\usepackage{graphicx}
\makeatletter
\def\maxwidth{\ifdim\Gin@nat@width>\linewidth\linewidth\else\Gin@nat@width\fi}
\def\maxheight{\ifdim\Gin@nat@height>\textheight\textheight\else\Gin@nat@height\fi}
\makeatother
% Scale images if necessary, so that they will not overflow the page
% margins by default, and it is still possible to overwrite the defaults
% using explicit options in \includegraphics[width, height, ...]{}
\setkeys{Gin}{width=\maxwidth,height=\maxheight,keepaspectratio}
% Set default figure placement to htbp
\makeatletter
\def\fps@figure{htbp}
\makeatother
\usepackage{svg}
\setlength{\emergencystretch}{3em} % prevent overfull lines
\providecommand{\tightlist}{%
  \setlength{\itemsep}{0pt}\setlength{\parskip}{0pt}}
\setcounter{secnumdepth}{-\maxdimen} % remove section numbering
\ifLuaTeX
  \usepackage{selnolig}  % disable illegal ligatures
\fi
\IfFileExists{bookmark.sty}{\usepackage{bookmark}}{\usepackage{hyperref}}
\IfFileExists{xurl.sty}{\usepackage{xurl}}{} % add URL line breaks if available
\urlstyle{same}
\hypersetup{
  pdftitle={fingR A support for Sediment Source Fingerprinting studies},
  pdfauthor={Thomas Chalaux-Clergue},
  hidelinks,
  pdfcreator={LaTeX via pandoc}}

\title{fingR A support for Sediment Source Fingerprinting studies}
\author{Thomas Chalaux-Clergue}
\date{}

\begin{document}
\maketitle

\hypertarget{fingr}{%
\section{\texorpdfstring{fingR }{fingR }}\label{fingr}}

\includegraphics{https://img.shields.io/github/r-package/v/tchalauxclergue/fingR?logo=github}
\includegraphics{https://img.shields.io/github/release-date/tchalauxclergue/fingR?color=blue}
\href{https://doi.org/10.5281/zenodo.10044404}{\includesvg{https://zenodo.org/badge/DOI/10.5281/zenodo.1402028.svg}}
\href{http://www.repostatus.org/\#active}{\includegraphics{https://img.shields.io/github/downloads/tchalauxclergue/fingR/total?style=flat}
\includesvg{http://www.repostatus.org/badges/latest/active.svg}}

\hypertarget{overview}{%
\subsection{Overview}\label{overview}}

\texttt{fingR} is a comprehensive package designed to support Sediment
Source Fingerprinting studies. It provides essentials tools including:
dataset characterisation, tracer selection from analysed properties
through the Three-step method, model source contributions modelling with
the Bayesian Mixing Model (BMM), and assessment of modelling predictions
prediction though the use of virtual mixtures, supporting BMM and
\href{http://brianstock.github.io/MixSIAR/index.html}{MixSIAR} models.

The \texttt{fingR} package is available in this
\href{https://github.com/tchalauxclergue/fingR}{Github} repository and
archived on \href{https://zenodo.org/records/10796375}{Zenodo}.

\hypertarget{table-of-content}{%
\subsubsection{Table of content}\label{table-of-content}}

\begin{itemize}
\tightlist
\item
  \protect\hyperlink{installation}{Installation}
\item
  \protect\hyperlink{usage}{Usage}

  \begin{itemize}
  \tightlist
  \item
    \protect\hyperlink{data-preparation}{Data preparation}
  \item
    \protect\hyperlink{tracer-selection}{Tracer selection}

    \begin{itemize}
    \tightlist
    \item
      \protect\hyperlink{1-assessment-of-conservative-behaviour}{1.
      Assessment of conservative behaviour}
    \item
      \protect\hyperlink{2-discriminant-power}{2. Discriminant power}
    \item
      \protect\hyperlink{selected-tracers}{Selected tracers}
    \item
      \protect\hyperlink{3-discriminant-function-analysis-dfa-stepwise-selection}{3.
      Discriminant Function Analysis (DFA) stepwise selection}
    \end{itemize}
  \item
    \protect\hyperlink{source-contribution-modelling}{Source
    contribution modelling}

    \begin{itemize}
    \tightlist
    \item
      \protect\hyperlink{virtual-mixtures}{Virtual mixtures}
    \item
      \protect\hyperlink{un-mixing-models}{Un-mixing models}

      \begin{itemize}
      \tightlist
      \item
        \protect\hyperlink{bayesian-mean-model-bmm}{Bayesian Mean Model
        (BMM)}

        \begin{itemize}
        \tightlist
        \item
          \protect\hyperlink{run-bmm-model-with-or-without_isotopic-ratio}{Run
          BMM model with or without isotopic ratio}
        \item
          \protect\hyperlink{modelling-accuracy-statistics}{Modelling
          accuarcy statistics}
        \end{itemize}
      \item
        \protect\hyperlink{mixsiar}{MixSIAR}

        \begin{itemize}
        \tightlist
        \item
          \protect\hyperlink{generate-data-for-mixsiar}{Generate data
          for MixSIAR}
        \item
          \protect\hyperlink{load-mixture-source-and-discrimination-data}{Load
          mixture, source and discrimination data}
        \item
          \protect\hyperlink{write-jags-model-file}{Write JAGS model
          file}
        \item
          \protect\hyperlink{run-mixsiar-model}{Run MixSIAR model}
        \item
          \protect\hyperlink{modelling-accuracy-statistics-1}{Modelling
          accuracy statistics}
        \end{itemize}
      \end{itemize}
    \end{itemize}
  \end{itemize}
\item
  \protect\hyperlink{future-updates}{Future updates}
\item
  \protect\hyperlink{getting-help}{Getting help}
\item
  \protect\hyperlink{citation}{Citation}
\item
  \protect\hyperlink{references}{References}
\end{itemize}

\hypertarget{installation}{%
\subsection{Installation}\label{installation}}

\begin{Shaded}
\begin{Highlighting}[]
\CommentTok{\#install.packages(devtools)}
\FunctionTok{library}\NormalTok{(devtools)}

\CommentTok{\# Install the lastest version from GitHub}
\NormalTok{devtools}\SpecialCharTok{::}\FunctionTok{install\_github}\NormalTok{(}\StringTok{"https://github.com/tchalauxclergue/fingR/releases/tag/2.0.0"}\NormalTok{, }\AttributeTok{ref =} \StringTok{"master"}\NormalTok{, }\AttributeTok{force =}\NormalTok{ T)}

\CommentTok{\# Alternatively, from the downloaded .tar.gz file}
\NormalTok{devtools}\SpecialCharTok{::}\FunctionTok{install\_local}\NormalTok{(}\StringTok{"path\_to\_file/fingR\_2.0.0.tar.gz"}\NormalTok{, }\AttributeTok{repos =} \ConstantTok{NULL}\NormalTok{) }\CommentTok{\# \textquotesingle{}path\_to\_file\textquotesingle{} should be modified accordingly to your working environment}
\end{Highlighting}
\end{Shaded}

\hypertarget{usage}{%
\subsection{Usage}\label{usage}}

\hypertarget{data-preparation}{%
\subsubsection{Data preparation}\label{data-preparation}}

To illustrate the usage of the package, we are using the database of the
sediment core sampled in the Mano Dam reservoir (Fukushima, Japan) and
associated soil samples. The \textbf{38} sediment core layer are used as
target, and \textbf{68} soil samples as potential sources. The potential
source include three classes: undecontaminated cropland (n =
\textbf{24}), remediated cropland (n = \textbf{22}), forest (n =
\textbf{24}), and subsoil (mainly granite saprolite; n = \textbf{24}).

All samples were sieved to 63 microns and analysed for organic matter,
elemental geochemistry and diffuse reflectance spectrocolourimetry for
sediment source fingerprinting.

The dataset, along with detailed measurement protocols, is available for
download on Zenodo at
\href{https://zenodo.org/doi/10.5281/zenodo.7081093}{Chalaux-Clergue et
al., 2024 (Version 2)}.

\begin{Shaded}
\begin{Highlighting}[]
\FunctionTok{library}\NormalTok{(fingR)}

\CommentTok{\# Get the dir to data and metadata files within the R package}
\NormalTok{data.dr }\OtherTok{\textless{}{-}} \FunctionTok{system.file}\NormalTok{(}\StringTok{"extdata"}\NormalTok{, }\StringTok{"TCC\_MDD\_20210608\_data\_ChalauxClergue\_et\_al\_v240319.csv"}\NormalTok{, }\AttributeTok{package =} \StringTok{"fingR"}\NormalTok{)}
\NormalTok{metadata.dr }\OtherTok{\textless{}{-}} \FunctionTok{system.file}\NormalTok{(}\StringTok{"extdata"}\NormalTok{, }\StringTok{"TCC\_MDD\_20210608\_metadata\_ChalauxClergue\_et\_al\_v240319.csv"}\NormalTok{, }\AttributeTok{package =} \StringTok{"fingR"}\NormalTok{)}

\CommentTok{\# Load the csv files of data and metadata {-} replace the dir with your file direction}
\NormalTok{db.data }\OtherTok{\textless{}{-}} \FunctionTok{read.csv}\NormalTok{(data.dr, }\AttributeTok{sep =} \StringTok{";"}\NormalTok{, }\AttributeTok{fileEncoding =} \StringTok{"latin1"}\NormalTok{, }\AttributeTok{na =} \StringTok{""}\NormalTok{)}
\NormalTok{db.metadata }\OtherTok{\textless{}{-}} \FunctionTok{read.csv}\NormalTok{(metadata.dr, }\AttributeTok{sep =} \StringTok{";"}\NormalTok{, }\AttributeTok{fileEncoding =} \StringTok{"latin1"}\NormalTok{, }\AttributeTok{na =} \StringTok{""}\NormalTok{)}
\end{Highlighting}
\end{Shaded}

Verify the different samples classes

\begin{Shaded}
\begin{Highlighting}[]
\FunctionTok{table}\NormalTok{(db.metadata}\SpecialCharTok{$}\NormalTok{Class\_decontamination)}
\CommentTok{\#\textgreater{} }
\CommentTok{\#\textgreater{}           Forest       Remediated          Subsoil           Target }
\CommentTok{\#\textgreater{}               24               10               10               38 }
\CommentTok{\#\textgreater{} Undecontaminated }
\CommentTok{\#\textgreater{}               24}
\end{Highlighting}
\end{Shaded}

We join the metadata (general information) and the data (analyses) so
that all the information is on a single dataframe. Both dataframes are
joined by common variables, here IGSN and Sample\_name. In addition,
only the analyses performed on the sample fraction below 63 microns are
kept.

\begin{Shaded}
\begin{Highlighting}[]
\FunctionTok{library}\NormalTok{(dplyr)}

\CommentTok{\# Create a single dataframe with metadata and data information}
\NormalTok{database }\OtherTok{\textless{}{-}}\NormalTok{ dplyr}\SpecialCharTok{::}\FunctionTok{left\_join}\NormalTok{(db.metadata, db.data, }\AttributeTok{by =} \FunctionTok{join\_by}\NormalTok{(IGSN, Sample\_name)) }\SpecialCharTok{\%\textgreater{}\%} \CommentTok{\# Joining metadata and data data frame}
\NormalTok{  dplyr}\SpecialCharTok{::}\FunctionTok{filter}\NormalTok{(Sample\_size }\SpecialCharTok{==} \StringTok{"\textless{} 63 µm"}\NormalTok{) }\SpecialCharTok{\%\textgreater{}\%} \CommentTok{\# select sample fraction on which analyses were performed}
\NormalTok{  dplyr}\SpecialCharTok{::}\FunctionTok{filter}\NormalTok{(Class\_decontamination }\SpecialCharTok{!=} \StringTok{"Remediated"}\NormalTok{) }\CommentTok{\# to simplify the example remediated cropland are removed}
\end{Highlighting}
\end{Shaded}

\begin{Shaded}
\begin{Highlighting}[]
\FunctionTok{table}\NormalTok{(database}\SpecialCharTok{$}\NormalTok{Class\_decontamination)}
\CommentTok{\#\textgreater{} }
\CommentTok{\#\textgreater{}           Forest          Subsoil           Target Undecontaminated }
\CommentTok{\#\textgreater{}               24               10               38               24}
\end{Highlighting}
\end{Shaded}

Among the analysed properties, 31 properties from organic matter and
elemental geochemistry analyses were selected as potential tracers.
Together with the properties, their measurement uncertainties are
selected.

\begin{Shaded}
\begin{Highlighting}[]
\CommentTok{\# colnames(database)}

\CommentTok{\# Select the names/colnames of the properties}
\NormalTok{prop.values }\OtherTok{\textless{}{-}}\NormalTok{ database }\SpecialCharTok{\%\textgreater{}\%}\NormalTok{ dplyr}\SpecialCharTok{::}\FunctionTok{select}\NormalTok{(TOC\_PrC, TN\_PrC,}\CommentTok{\# organic matter properties}
\NormalTok{                                          EDXRF\_Al\_mg.kg}\FloatTok{.1}\SpecialCharTok{:}\NormalTok{EDXRF\_Zr\_mg.kg}\FloatTok{.1}\NormalTok{) }\SpecialCharTok{\%\textgreater{}\%}\NormalTok{ names }\CommentTok{\# elemental geochemistry}


\CommentTok{\# Select the names/colnames of the property measurement uncertainties/errors}
\NormalTok{prop.uncertainties }\OtherTok{\textless{}{-}}\NormalTok{ database }\SpecialCharTok{\%\textgreater{}\%}\NormalTok{ dplyr}\SpecialCharTok{::}\FunctionTok{select}\NormalTok{(TOC\_SD, TN\_SD, }\CommentTok{\# organic matter}
\NormalTok{                                                 EDXRF\_Al\_RMSE}\SpecialCharTok{:}\NormalTok{EDXRF\_Zr\_RMSE) }\SpecialCharTok{\%\textgreater{}\%}\NormalTok{ names }\CommentTok{\# elemental geochemistry}

\FunctionTok{names}\NormalTok{(prop.uncertainties) }\OtherTok{\textless{}{-}}\NormalTok{ prop.values }\CommentTok{\# Add property names to property uncertainty for easier selection}
\end{Highlighting}
\end{Shaded}

\begin{Shaded}
\begin{Highlighting}[]
\NormalTok{prop.values}
\CommentTok{\#\textgreater{}  [1] "TOC\_PrC"          "TN\_PrC"           "EDXRF\_Al\_mg.kg.1" "EDXRF\_Ca\_mg.kg.1"}
\CommentTok{\#\textgreater{}  [5] "EDXRF\_Co\_mg.kg.1" "EDXRF\_Cr\_mg.kg.1" "EDXRF\_Cu\_mg.kg.1" "EDXRF\_Fe\_mg.kg.1"}
\CommentTok{\#\textgreater{}  [9] "EDXRF\_K\_mg.kg.1"  "EDXRF\_Mg\_mg.kg.1" "EDXRF\_Mn\_mg.kg.1" "EDXRF\_Ni\_mg.kg.1"}
\CommentTok{\#\textgreater{} [13] "EDXRF\_Pb\_mg.kg.1" "EDXRF\_Rb\_mg.kg.1" "EDXRF\_Si\_mg.kg.1" "EDXRF\_Sr\_mg.kg.1"}
\CommentTok{\#\textgreater{} [17] "EDXRF\_Ti\_mg.kg.1" "EDXRF\_Zn\_mg.kg.1" "EDXRF\_Zr\_mg.kg.1"}

\FunctionTok{unname}\NormalTok{(prop.uncertainties)}
\CommentTok{\#\textgreater{}  [1] "TOC\_SD"        "TN\_SD"         "EDXRF\_Al\_RMSE" "EDXRF\_Ca\_RMSE"}
\CommentTok{\#\textgreater{}  [5] "EDXRF\_Co\_RMSE" "EDXRF\_Cr\_RMSE" "EDXRF\_Cu\_RMSE" "EDXRF\_Fe\_RMSE"}
\CommentTok{\#\textgreater{}  [9] "EDXRF\_K\_RMSE"  "EDXRF\_Mg\_RMSE" "EDXRF\_Mn\_RMSE" "EDXRF\_Ni\_RMSE"}
\CommentTok{\#\textgreater{} [13] "EDXRF\_Pb\_RMSE" "EDXRF\_Rb\_RMSE" "EDXRF\_Si\_RMSE" "EDXRF\_Sr\_RMSE"}
\CommentTok{\#\textgreater{} [17] "EDXRF\_Ti\_RMSE" "EDXRF\_Zn\_RMSE" "EDXRF\_Zr\_RMSE"}
\end{Highlighting}
\end{Shaded}

First, we use \texttt{data.watcher} to check that the selected
properties meet the quality criteria, particularly in terms of their
measurement uncertainty. Several criteria are evaluated (e.g.~presence
of some negative values or high uncertainty) and presented as indicators
to consider the use of a property.

\begin{Shaded}
\begin{Highlighting}[]
\FunctionTok{library}\NormalTok{(fingR)}

\NormalTok{fingR}\SpecialCharTok{::}\FunctionTok{data.watcher}\NormalTok{(}\AttributeTok{data =}\NormalTok{ database, }\AttributeTok{properties =}\NormalTok{ prop.values, }\AttributeTok{prop.uncer =}\NormalTok{ prop.uncertainties)}
\CommentTok{\#\textgreater{} }
\CommentTok{\#\textgreater{} Following column(s) contain(s) some negative values: EDXRF\_Cr\_mg.kg.1.}
\CommentTok{\#\textgreater{} Following column(s) have a measurement uncertainty that makes some values to be virtually impossible: EDXRF\_Co\_mg.kg.1, EDXRF\_Cr\_mg.kg.1, EDXRF\_Cu\_mg.kg.1, EDXRF\_Ni\_mg.kg.1.}
\CommentTok{\#\textgreater{} Following column(s) have a relative measurement uncertainty above 5\% (up to {-} number): EDXRF\_Co\_mg.kg.1 (max:753\% {-} n:26), EDXRF\_Cr\_mg.kg.1 (max:211\% {-} n:38), EDXRF\_Ni\_mg.kg.1 (max:105\% {-} n:96), EDXRF\_Cu\_mg.kg.1 (max:103\% {-} n:52), EDXRF\_Rb\_mg.kg.1 (max:89\% {-} n:93), TN\_PrC (max:45\% {-} n:91), EDXRF\_Pb\_mg.kg.1 (max:38\% {-} n:91), EDXRF\_Zn\_mg.kg.1 (max:34\% {-} n:96), EDXRF\_Sr\_mg.kg.1 (max:15\% {-} n:46), TOC\_PrC (max:14\% {-} n:95), EDXRF\_Zr\_mg.kg.1 (max:7\% {-} n:2).}
\end{Highlighting}
\end{Shaded}

According to \texttt{data.watcher} results: Co, Cr, Cu, Ni, and Rb have
too high measurement uncertainty and in addition Cr has some negative
values among the samples. These properties will be removed from
following study.

\begin{Shaded}
\begin{Highlighting}[]
\CommentTok{\# Remove Co, Cr, Cu, Ni and Rb from the vector of properties}
\NormalTok{prop.values }\OtherTok{\textless{}{-}}\NormalTok{ prop.values[}\SpecialCharTok{!}\NormalTok{prop.values }\SpecialCharTok{\%in\%} \FunctionTok{c}\NormalTok{(}\StringTok{"EDXRF\_Co\_mg.kg.1"}\NormalTok{, }\StringTok{"EDXRF\_Cr\_mg.kg.1"}\NormalTok{, }\StringTok{"EDXRF\_Cu\_mg.kg.1"}\NormalTok{, }\StringTok{"EDXRF\_Ni\_mg.kg.1"}\NormalTok{, }\StringTok{"EDXRF\_Rb\_mg.kg.1"}\NormalTok{)]}

\CommentTok{\# Keep uncertainties associated to the new vector of properties}
\NormalTok{prop.uncertainties }\OtherTok{\textless{}{-}}\NormalTok{ prop.uncertainties[prop.values]}
\end{Highlighting}
\end{Shaded}

\begin{Shaded}
\begin{Highlighting}[]
\NormalTok{prop.values}
\CommentTok{\#\textgreater{}  [1] "TOC\_PrC"          "TN\_PrC"           "EDXRF\_Al\_mg.kg.1" "EDXRF\_Ca\_mg.kg.1"}
\CommentTok{\#\textgreater{}  [5] "EDXRF\_Fe\_mg.kg.1" "EDXRF\_K\_mg.kg.1"  "EDXRF\_Mg\_mg.kg.1" "EDXRF\_Mn\_mg.kg.1"}
\CommentTok{\#\textgreater{}  [9] "EDXRF\_Pb\_mg.kg.1" "EDXRF\_Si\_mg.kg.1" "EDXRF\_Sr\_mg.kg.1" "EDXRF\_Ti\_mg.kg.1"}
\CommentTok{\#\textgreater{} [13] "EDXRF\_Zn\_mg.kg.1" "EDXRF\_Zr\_mg.kg.1"}
\end{Highlighting}
\end{Shaded}

\hypertarget{tracer-selection}{%
\subsubsection{Tracer selection}\label{tracer-selection}}

\hypertarget{assessment-of-conservative-behaviour}{%
\subsubsection{1. Assessment of conservative
behaviour}\label{assessment-of-conservative-behaviour}}

In the three-step method, the conservative behaviour is assessed by
range tests (RT), also known as bracket tests. To be considered to have
a conservative behaviour, all target samples values should lye within
the range of the potential source classes. The range of the potential
source classes is defined as the highest and lowest source class value
of a certain criterion.

Various criteria for range tests are documented in the literature,
including minimum-maximum (\textbf{MM}), minimum-maximum plus/minus 10\%
(\textbf{MMe}) -to account for measurement error- , boxplot
\textbf{whiskers} -as threshold to identify extreme values-, boxplot
\textbf{hinge} -50\% of the population-, \textbf{mean}, mean plus/minus
one standard deviation (\textbf{mean.sd}) and median. The \textbf{mean}
and \textbf{mean.sd} criteria are performed on log-transformed values,
assuming a Normal distribution of the samples.

By default, the function applies all these criteria, though their
effectiveness in identifying conservative characteristics may vary.
Among these, the \textbf{mean.sd} criterion is mathematically the most
robust.

The \texttt{range.test} function returns a list containing two data
frames - \emph{results.df}: A summary overview of the range test
results. - \emph{results.RT}: Detailed results for each target sample's
range test for each property.

\begin{Shaded}
\begin{Highlighting}[]
\NormalTok{rt.results }\OtherTok{\textless{}{-}}\NormalTok{ fingR}\SpecialCharTok{::}\FunctionTok{range.tests}\NormalTok{(}\AttributeTok{data =}\NormalTok{ database,                 }\CommentTok{\# Dataset containing source and mixture information}
                                 \AttributeTok{class =} \StringTok{"Class\_decontamination"}\NormalTok{, }\CommentTok{\# Column containing the classification or grouping of sources and mixtures}
                                 \AttributeTok{mixture =} \StringTok{"Target"}\NormalTok{,              }\CommentTok{\# Identifier for mixtures within the class variable}
                                 \AttributeTok{properties =}\NormalTok{ prop.values,        }\CommentTok{\# Properties to be tested for conservativeness}
                                 \AttributeTok{sample.id =} \StringTok{"Sample\_name"}\NormalTok{,       }\CommentTok{\# Identifier for individual samples}
                                 \AttributeTok{criteria =} \FunctionTok{c}\NormalTok{(}\StringTok{"mean.sd"}\NormalTok{)          }\CommentTok{\# Criteria for conducting range tests (options: "MM", "MMe", "whiskers", "hinge", "mean", "mean.sd", "median", or "all")}
                                 \CommentTok{\# MM.error = c(0.1),             \# Optional: Set the minimum{-}maximum plus/minus error as 10\%}
                                 \CommentTok{\# save.dir = dir.example,        \# Optional: Directory path for saving the results}
                                 \CommentTok{\# note = "example"               \# Optional: Additional note to append to the file name}
\NormalTok{                                 )}
\end{Highlighting}
\end{Shaded}

\begin{Shaded}
\begin{Highlighting}[]
\NormalTok{rt.results}\SpecialCharTok{$}\NormalTok{results.df[}\DecValTok{1}\SpecialCharTok{:}\DecValTok{5}\NormalTok{]}
\CommentTok{\#\textgreater{}            Property n\_source n\_mixture NAs RT\_mean.sd\_single}
\CommentTok{\#\textgreater{} 1           TOC\_PrC       58        38   0              TRUE}
\CommentTok{\#\textgreater{} 2            TN\_PrC       58        38   0              TRUE}
\CommentTok{\#\textgreater{} 3  EDXRF\_Al\_mg.kg.1       58        38   0              TRUE}
\CommentTok{\#\textgreater{} 4  EDXRF\_Ca\_mg.kg.1       58        38   0             FALSE}
\CommentTok{\#\textgreater{} 5  EDXRF\_Fe\_mg.kg.1       58        38   0             FALSE}
\CommentTok{\#\textgreater{} 6   EDXRF\_K\_mg.kg.1       58        38   0             FALSE}
\CommentTok{\#\textgreater{} 7  EDXRF\_Mg\_mg.kg.1       58        38   0             FALSE}
\CommentTok{\#\textgreater{} 8  EDXRF\_Mn\_mg.kg.1       58        38   0             FALSE}
\CommentTok{\#\textgreater{} 9  EDXRF\_Pb\_mg.kg.1       58        38   0             FALSE}
\CommentTok{\#\textgreater{} 10 EDXRF\_Si\_mg.kg.1       58        38   0             FALSE}
\CommentTok{\#\textgreater{} 11 EDXRF\_Sr\_mg.kg.1       58        38   0             FALSE}
\CommentTok{\#\textgreater{} 12 EDXRF\_Ti\_mg.kg.1       58        38   0              TRUE}
\CommentTok{\#\textgreater{} 13 EDXRF\_Zn\_mg.kg.1       58        38   0             FALSE}
\CommentTok{\#\textgreater{} 14 EDXRF\_Zr\_mg.kg.1       58        38   0             FALSE}
\end{Highlighting}
\end{Shaded}

The \texttt{is.conservative} function returns a list of vector of
conservative properties based on the results of range tests. If multiple
criteria are used, a vector is generated for each criterion.

\begin{Shaded}
\begin{Highlighting}[]
\NormalTok{prop.cons }\OtherTok{\textless{}{-}}\NormalTok{ fingR}\SpecialCharTok{::}\FunctionTok{is.conservative}\NormalTok{(}\AttributeTok{data =}\NormalTok{ rt.results}\SpecialCharTok{$}\NormalTok{results.df,  }\CommentTok{\# Data frame containing the results of range tests, typically generated by fingR::range.tests}
                                    \CommentTok{\# property = "Property",       \# Optional: Column containing the names of properties being tested for conservativeness}
                                    \CommentTok{\# test.format = "RT",          \# Optional: Indicates the common pattern in column test names (default: "RT")}
                                    \CommentTok{\# position = 2,                \# Optional: Position of the test name in the column name (default: 2)}
                                    \CommentTok{\# separator = "\_",             \# Optional: Character used to split test names in the column (default: "\_")}
                                    \CommentTok{\# note = "example"             \# Optional: Additional note to append to the file name}
\NormalTok{                                    )}
\end{Highlighting}
\end{Shaded}

\begin{Shaded}
\begin{Highlighting}[]
\NormalTok{prop.cons}
\CommentTok{\#\textgreater{} $mean.sd}
\CommentTok{\#\textgreater{} [1] "TOC\_PrC"          "TN\_PrC"           "EDXRF\_Al\_mg.kg.1" "EDXRF\_Ti\_mg.kg.1"}
\end{Highlighting}
\end{Shaded}

\hypertarget{discriminant-power}{%
\subsubsection{2. Discriminant power}\label{discriminant-power}}

Inthe three-step method, the capacity of a property to discriminate
among source groups is commonly assessed using a Kruskal-Wallis H-test.
The \emph{discriminant.test} function arguments are very similar to
\emph{range.tests}. As an alternative Kolmogov-Smirnov two-samples tests
can be used. It provides more detailled results as source groups are
compared to each other.

\begin{Shaded}
\begin{Highlighting}[]
\NormalTok{KS.results }\OtherTok{\textless{}{-}}\NormalTok{ fingR}\SpecialCharTok{::}\FunctionTok{discriminant.test}\NormalTok{(}\AttributeTok{data =}\NormalTok{ database,                 }\CommentTok{\# Dataset containing source and mixture information}
                                       \AttributeTok{class =} \StringTok{"Class\_decontamination"}\NormalTok{, }\CommentTok{\# Column containing the classification or grouping of sources and mixtures}
                                       \AttributeTok{mixture =} \StringTok{"Target"}\NormalTok{,              }\CommentTok{\# Identifier for mixtures within the class variable}
                                       \AttributeTok{test =} \StringTok{"KS"}\NormalTok{,                     }\CommentTok{\# Type of test performed, Kruskal{-}Wallis (KW) or Kolmogorov{-}smirnov (KS)}
                                       \AttributeTok{properties =}\NormalTok{ prop.values,        }\CommentTok{\# Properties to be tested for conservativeness}
                                       \AttributeTok{p.level =}\NormalTok{ .}\DecValTok{01}\NormalTok{,                   }\CommentTok{\# Optional: p{-}value significance level (default = 0.05)}
                                       \CommentTok{\# save.discrim.tests = T,        \# Optional: If two{-}samples tests should be saved}
                                       \CommentTok{\# save.dir = dir.example,        \# Optional: Directory path for saving the results}
                                       \CommentTok{\# note = "example"               \# Optional: Additional note to append to the file name}
\NormalTok{                                       )}
\end{Highlighting}
\end{Shaded}

\begin{Shaded}
\begin{Highlighting}[]
\NormalTok{KS.results[}\DecValTok{1}\SpecialCharTok{:}\DecValTok{5}\NormalTok{,]}
\CommentTok{\#\textgreater{}           Property n.diff.groups Kolmogorov.Smirnov\_discriminant}
\CommentTok{\#\textgreater{} 1          TOC\_PrC             3                            TRUE}
\CommentTok{\#\textgreater{} 2           TN\_PrC             3                            TRUE}
\CommentTok{\#\textgreater{} 3 EDXRF\_Al\_mg.kg.1             3                            TRUE}
\CommentTok{\#\textgreater{} 4 EDXRF\_Ca\_mg.kg.1             1                            TRUE}
\CommentTok{\#\textgreater{} 5 EDXRF\_Fe\_mg.kg.1             0                           FALSE}
\end{Highlighting}
\end{Shaded}

Properties that get a Kruskal-Wallis p-value bellow 0.05
(\textbf{p.value = 0.05}), are selected as discriminant properties. The
function \emph{is.discriminant} list them. The function automatically
recognise data.frame produced by \emph{discriminant.test} but it is
possible to set it for other data.frame format.

\begin{Shaded}
\begin{Highlighting}[]
\NormalTok{prop.discrim }\OtherTok{\textless{}{-}}\NormalTok{ fingR}\SpecialCharTok{::}\FunctionTok{is.discriminant}\NormalTok{(KS.results,                                }\CommentTok{\# data.frame from discriminant.test or any df with the same organisation.}
                                       \CommentTok{\# property = "Property",                   \# Optional: Column containing the names of properties being tested for conservativeness}
                                       \CommentTok{\# test.format = "Kruskal.Wallis\_p.value",  \# Optional: Indicates the common pattern in column test names (default: "RT")}
                                       \CommentTok{\# position = 1,                            \# Optional: Position of the test name in the column name (default: 1)}
                                       \CommentTok{\# separator = "\_",                         \# Optional: Character used to split test names in the column (default: "\_")}
                                       \CommentTok{\# p.level = 0.05,                          \# Optional: p{-}value significance level (default = 0.05)}
                                       \CommentTok{\# note = "example"                         \# Optional: Additional note to append to the file name}
\NormalTok{                                       )}
\end{Highlighting}
\end{Shaded}

\begin{Shaded}
\begin{Highlighting}[]
\NormalTok{prop.discrim}
\CommentTok{\#\textgreater{} $Kolmogorov.Smirnov}
\CommentTok{\#\textgreater{}  [1] "TOC\_PrC"          "TN\_PrC"           "EDXRF\_Al\_mg.kg.1" "EDXRF\_Ca\_mg.kg.1"}
\CommentTok{\#\textgreater{}  [5] "EDXRF\_K\_mg.kg.1"  "EDXRF\_Pb\_mg.kg.1" "EDXRF\_Si\_mg.kg.1" "EDXRF\_Sr\_mg.kg.1"}
\CommentTok{\#\textgreater{}  [9] "EDXRF\_Zn\_mg.kg.1" "EDXRF\_Zr\_mg.kg.1"}
\end{Highlighting}
\end{Shaded}

\hypertarget{selected-tracers}{%
\paragraph{Selected tracers}\label{selected-tracers}}

Tracers are conservative and discriminant properties.

\begin{Shaded}
\begin{Highlighting}[]
\NormalTok{tracers }\OtherTok{\textless{}{-}}\NormalTok{ fingR}\SpecialCharTok{::}\FunctionTok{selected.tracers}\NormalTok{(}\AttributeTok{cons =}\NormalTok{ prop.cons,        }\CommentTok{\# character vector of conservative properties}
                                   \AttributeTok{discrim =}\NormalTok{ prop.discrim)  }\CommentTok{\# character Vector of discriminant properties}
\end{Highlighting}
\end{Shaded}

\begin{Shaded}
\begin{Highlighting}[]
\NormalTok{tracers }
\CommentTok{\#\textgreater{} $mean.sd\_Kolmogorov.Smirnov}
\CommentTok{\#\textgreater{} [1] "TOC\_PrC"          "TN\_PrC"           "EDXRF\_Al\_mg.kg.1"}
\end{Highlighting}
\end{Shaded}

Tracer selection are labelled by \texttt{selected.tracers} accordingly
to the range test criteria (e.g.~mean.sd, hinge\ldots) and discriminant
test (i.e.~Kruskal.Wallis or Kolmogorov.Smirnov). However, sometimes
this label is to long for file labelling therefore, you may replace it
accondingly.

\begin{Shaded}
\begin{Highlighting}[]
\FunctionTok{names}\NormalTok{(tracers) }\OtherTok{\textless{}{-}} \StringTok{"msd\_KS"} \CommentTok{\# replace tracers names with the new name}
\end{Highlighting}
\end{Shaded}

\begin{Shaded}
\begin{Highlighting}[]
\NormalTok{tracers}
\CommentTok{\#\textgreater{} $msd\_KS}
\CommentTok{\#\textgreater{} [1] "TOC\_PrC"          "TN\_PrC"           "EDXRF\_Al\_mg.kg.1"}
\end{Highlighting}
\end{Shaded}

\hypertarget{discriminant-function-analysis-dfa-stepwise-selection}{%
\subsubsection{3. Discriminant Function Analysis (DFA) stepwise
selection}\label{discriminant-function-analysis-dfa-stepwise-selection}}

The conventional three-step method apply a DFA forward stepwise
selection on the selected tracers. This DFA stepwise selection aims to
retain tracers that maximize source discrimination. However, this step
has faced criticism. Observing the results for a large selection of
tracers can be insightful. However, it is not useful for small selection
of tracers, as it is the case here.

\begin{Shaded}
\begin{Highlighting}[]
\NormalTok{tracers.SW }\OtherTok{\textless{}{-}}\NormalTok{ fingR}\SpecialCharTok{::}\FunctionTok{stepwise.selection}\NormalTok{(}\AttributeTok{data =}\NormalTok{ database,                 }\CommentTok{\# Dataset containing source and mixture information}
                                        \AttributeTok{class =} \StringTok{"Class\_decontamination"}\NormalTok{, }\CommentTok{\# Column containing the classification or grouping of source and mixtures}
                                        \AttributeTok{tracers =}\NormalTok{ tracers}\SpecialCharTok{$}\NormalTok{msd\_KS,        }\CommentTok{\# Character vector containing tracers to consider}
                                        \AttributeTok{target =} \StringTok{"Target"}                \CommentTok{\# Identifier for target samples within the "class" column}
                                        \CommentTok{\# save.dir = dir.example,        \# Optional: Directory path for saving the results}
                                        \CommentTok{\# note = "example"               \# Optional: Additional note to append to the file name}
\NormalTok{                                        )}
\end{Highlighting}
\end{Shaded}

\begin{Shaded}
\begin{Highlighting}[]
\NormalTok{tracers.SW}
\CommentTok{\#\textgreater{} [1] "EDXRF\_Al\_mg.kg.1" "TOC\_PrC"          "TN\_PrC"}
\end{Highlighting}
\end{Shaded}

The DFA stepwise selection did not removed any of the selected tracers.
If the DFA selects different tracers, examining the modelling results
for both sets can provide useful insights.

Both tracer selections could joint like following:

\begin{Shaded}
\begin{Highlighting}[]
\CommentTok{\# Joining two tracers vector in a list}
\NormalTok{all.tracers }\OtherTok{\textless{}{-}} \FunctionTok{list}\NormalTok{(}\StringTok{"msd\_KS"} \OtherTok{=}\NormalTok{ tracers}\SpecialCharTok{$}\NormalTok{msd\_KS, }\StringTok{"msd\_KS\_DFA"} \OtherTok{=}\NormalTok{ tracers.SW)}
\end{Highlighting}
\end{Shaded}

\hypertarget{source-contribution-modelling}{%
\subsection{Source contribution
modelling}\label{source-contribution-modelling}}

\hypertarget{virtual-mixtures}{%
\paragraph{Virtual mixtures}\label{virtual-mixtures}}

To evaluate the accuracy of un-mixing models, virtual mixtures are used.
These virtual mixtures, serving as target samples with known
contributions, allow for the calculation of modelling accuracy metrics.
The \texttt{VM.contrib.generator} generate virtual mixture contributions
from the \texttt{min} to the \texttt{max} contribution set with a
specified \texttt{step}. Contribution could be set as percentage
(\texttt{min\ =\ 0,\ max\ =\ 100}) or as a ratios
(\texttt{min\ =\ 0,\ max\ =\ 1}). Smaller \texttt{step} result in a
higher number of virtual mixtures, such as 231 virtual mixtures for a
5\% step and 5151 virtual mixtures for a 1\% step. Alternatively,
virtual mixtures can be generated within \texttt{VM.builder}.

\begin{Shaded}
\begin{Highlighting}[]
\CommentTok{\# Generate virtual mixture source contributions}
\NormalTok{VM.contrib }\OtherTok{\textless{}{-}}\NormalTok{ fingR}\SpecialCharTok{::}\FunctionTok{VM.contrib.generator}\NormalTok{(}\AttributeTok{n.sources =} \DecValTok{3}\NormalTok{,                                              }\CommentTok{\# Number of source levels}
                                          \AttributeTok{min =} \DecValTok{0}\NormalTok{,                                                    }\CommentTok{\# Minimum contribution (here percentage)}
                                          \AttributeTok{max =} \DecValTok{100}\NormalTok{,                                                  }\CommentTok{\# Maximum contribution (here percentage)}
                                          \AttributeTok{step =} \DecValTok{5}\NormalTok{,                                                   }\CommentTok{\# Step between two contribution levels (here percentage)}
                                          \AttributeTok{sources.class =} \FunctionTok{c}\NormalTok{(}\StringTok{"Forest"}\NormalTok{, }\StringTok{"Subsoil"}\NormalTok{, }\StringTok{"Undecontaminated"}\NormalTok{), }\CommentTok{\# Optional: Classification of sources}
                                          \AttributeTok{save.dir =}\NormalTok{ dir.example,                                     }\CommentTok{\# Optional: Directory path for saving the results}
                                          \CommentTok{\# note = "example"                                          \# Optional: Additional note to append to the file name}
                                          \CommentTok{\# VM.name = "Sample\_name",                                  \# Optional: Name of the column containing virtual mixture labels}
                                          \CommentTok{\# fileEncoding = "latin1",                                  \# Optional: File encoding, important if special character are used in source levels}
                                          \CommentTok{\# return = TRUE,                                            \# Optional: Whether the function should return the result}
                                          \CommentTok{\# save = TRUE                                               \# Optional: Whether the function should save the result}
\NormalTok{                                          )}
\end{Highlighting}
\end{Shaded}

\begin{Shaded}
\begin{Highlighting}[]
\NormalTok{VM.contrib[}\DecValTok{1}\SpecialCharTok{:}\DecValTok{5}\NormalTok{,]}
\CommentTok{\#\textgreater{}   Sample\_name Forest Subsoil Undecontaminated}
\CommentTok{\#\textgreater{} 1      VM{-}001      0       0              100}
\CommentTok{\#\textgreater{} 2      VM{-}002      0       5               95}
\CommentTok{\#\textgreater{} 3      VM{-}003      0      10               90}
\CommentTok{\#\textgreater{} 4      VM{-}004      0      15               85}
\CommentTok{\#\textgreater{} 5      VM{-}005      0      20               80}
\end{Highlighting}
\end{Shaded}

Next, virtual mixture properties are calculated as simple proportional
mixture of source signature (i.e.~mean values). This approach is a
simple mass balance approach. The \texttt{VM.builder} function saves and
returns a list containing three \emph{data.frame} objects: one with the
\texttt{\$property} values, the other with the \texttt{\$uncertainty}
values (with corresponding labels when given in \texttt{\$uncertainty}
if not simply ``\_SD'' is added at the end of the tracer label), and the
last one \texttt{\$full} where property and uncertainty were join.

To run un-mixing models, source and target information should be within
the same data frame. Source informations are added at the end of all the
\emph{data.frame} created.

\begin{Shaded}
\begin{Highlighting}[]
\NormalTok{VM }\OtherTok{\textless{}{-}} \FunctionTok{VM.builder}\NormalTok{(}\AttributeTok{data =}\NormalTok{ database,                                          }\CommentTok{\# Dataset containing source samples}
                 \AttributeTok{material =} \StringTok{"Material"}\NormalTok{,                                    }\CommentTok{\# Column indicating the difference between source and target}
                 \AttributeTok{source.name =} \StringTok{"Source"}\NormalTok{,                                   }\CommentTok{\# Identifier for source samples within the material column}
                 \AttributeTok{class =} \StringTok{"Class\_decontamination"}\NormalTok{,                          }\CommentTok{\# Column containing the classification or grouping of sources and mixtures}
                 \AttributeTok{tracers =}\NormalTok{ tracers}\SpecialCharTok{$}\NormalTok{msd\_KS,                                 }\CommentTok{\# Character vector containing tracers to consider}
                 \AttributeTok{uncertainty =} \FunctionTok{unname}\NormalTok{(prop.uncertainties[tracers}\SpecialCharTok{$}\NormalTok{msd\_KS]), }\CommentTok{\# Character vector containing tracers uncertainty labels}
                 \AttributeTok{contributions =}\NormalTok{ VM.contrib,                               }\CommentTok{\# Virtual mixture contributions}
                 \AttributeTok{VM.name =} \StringTok{"Sample\_name"}\NormalTok{,                                  }\CommentTok{\# Column with virtual mixture labels in the \textquotesingle{}contribution\textquotesingle{} (i.e. VM.contribution)}
                 \AttributeTok{add.sources =} \ConstantTok{TRUE}\NormalTok{,                                       }\CommentTok{\# Add source information at the end of the VM data frames}
                 \AttributeTok{save.dir =}\NormalTok{ dir.example,                                   }\CommentTok{\# Optional: Directory path for saving the results}
                 \CommentTok{\# note = "example"                                        \# Optional: Additional note to append to the file name}
\NormalTok{                 )}
\end{Highlighting}
\end{Shaded}

\begin{Shaded}
\begin{Highlighting}[]
\NormalTok{VM}\SpecialCharTok{$}\NormalTok{full[}\DecValTok{1}\SpecialCharTok{:}\DecValTok{5}\NormalTok{,]}
\CommentTok{\#\textgreater{}   Sample\_name Class\_decontamination TOC\_PrC TN\_PrC EDXRF\_Al\_mg.kg.1 TOC\_SD}
\CommentTok{\#\textgreater{} 1      VM{-}001       Virtual Mixture    5.16   0.42         84858.53   4.75}
\CommentTok{\#\textgreater{} 2      VM{-}002       Virtual Mixture    4.97   0.40         86004.71   4.75}
\CommentTok{\#\textgreater{} 3      VM{-}003       Virtual Mixture    4.78   0.39         87150.90   4.75}
\CommentTok{\#\textgreater{} 4      VM{-}004       Virtual Mixture    4.60   0.38         88297.08   4.75}
\CommentTok{\#\textgreater{} 5      VM{-}005       Virtual Mixture    4.41   0.36         89443.26   4.75}
\CommentTok{\#\textgreater{}   TN\_SD EDXRF\_Al\_RMSE}
\CommentTok{\#\textgreater{} 1  0.28      17840.72}
\CommentTok{\#\textgreater{} 2  0.28      17840.72}
\CommentTok{\#\textgreater{} 3  0.28      17840.72}
\CommentTok{\#\textgreater{} 4  0.28      17840.72}
\CommentTok{\#\textgreater{} 5  0.28      17840.72}
\end{Highlighting}
\end{Shaded}

Here an example of sets to generate virtual mixture with the
\texttt{VM.builder} function without previously running the
\texttt{VM.contrib.generator} function.

\begin{Shaded}
\begin{Highlighting}[]
\NormalTok{VM }\OtherTok{\textless{}{-}} \FunctionTok{VM.builder}\NormalTok{(}\AttributeTok{data =}\NormalTok{ database,                                          }\CommentTok{\# Dataset containing source samples}
                 \AttributeTok{material =} \StringTok{"Material"}\NormalTok{,                                    }\CommentTok{\# Column indicating the difference between source and target}
                 \AttributeTok{source.name =} \StringTok{"Source"}\NormalTok{,                                   }\CommentTok{\# Identifier for source samples within the material column}
                 \AttributeTok{class =} \StringTok{"Class\_decontamination"}\NormalTok{,                          }\CommentTok{\# Column containing the classification or grouping of sources and mixtures}
                 \AttributeTok{tracers =}\NormalTok{ tracers}\SpecialCharTok{$}\NormalTok{msd\_KS,                                 }\CommentTok{\# Character vector containing tracers to consider}
                 \AttributeTok{uncertainty =} \FunctionTok{unname}\NormalTok{(prop.uncertainties[tracers}\SpecialCharTok{$}\NormalTok{msd\_KS]), }\CommentTok{\# Character vector containing tracers uncertainty labels}
                 \AttributeTok{VM.range =} \FunctionTok{c}\NormalTok{(}\DecValTok{0}\NormalTok{, }\DecValTok{100}\NormalTok{),                                     }\CommentTok{\# Minimum and maximum contribution (here percentage)}
                 \AttributeTok{VM.step =} \DecValTok{5}\NormalTok{,                                              }\CommentTok{\# Step between two contribution levels (here percentage)}
                 \AttributeTok{VM.name =} \StringTok{"Sample\_name"}\NormalTok{,                                  }\CommentTok{\# Column with virtual mixture labels in the \textquotesingle{}contribution\textquotesingle{} (i.e. VM.contribution)}
                 \AttributeTok{add.sources =} \ConstantTok{TRUE}\NormalTok{,                                       }\CommentTok{\# Add source information at the end of the VM data frames}
                 \AttributeTok{save.dir =}\NormalTok{ dir.example,                                   }\CommentTok{\# Optional: Directory path for saving the results}
                 \CommentTok{\# note = "example"                                        \# Optional: Additional note to append to the file name}
\NormalTok{                 )}
\end{Highlighting}
\end{Shaded}

\hypertarget{un-mixing-models}{%
\subsubsection{Un-mixing models}\label{un-mixing-models}}

Create a folder where all modelling results will be saved

\begin{Shaded}
\begin{Highlighting}[]
\CommentTok{\# Create new folder to save tracer modelling results}
\FunctionTok{dir.create}\NormalTok{(}\FunctionTok{file.path}\NormalTok{(dir.example, }\StringTok{"Modelling/"}\NormalTok{), }\AttributeTok{showWarnings =} \ConstantTok{FALSE}\NormalTok{)}
\NormalTok{dir.modelling }\OtherTok{\textless{}{-}} \FunctionTok{paste0}\NormalTok{(dir.example, }\StringTok{"Modelling/"}\NormalTok{)}
\end{Highlighting}
\end{Shaded}

\hypertarget{bayesian-mean-model-bmm}{%
\subsubsection{Bayesian Mean Model
(BMM)}\label{bayesian-mean-model-bmm}}

Create a folder specific from BMM modelling results.

\begin{Shaded}
\begin{Highlighting}[]
\CommentTok{\# Create new folder to save BMM modelling results}
\FunctionTok{dir.create}\NormalTok{(}\FunctionTok{file.path}\NormalTok{(dir.modelling, }\StringTok{"BMM/"}\NormalTok{), }\AttributeTok{showWarnings =} \ConstantTok{FALSE}\NormalTok{)}
\NormalTok{dir.mod.BMM }\OtherTok{\textless{}{-}} \FunctionTok{paste0}\NormalTok{(dir.modelling, }\StringTok{"BMM/"}\NormalTok{)}
\end{Highlighting}
\end{Shaded}

\hypertarget{run-bmm-model-with-or-without-isotopic-ratio}{%
\paragraph{Run BMM model with or without isotopic
ratio}\label{run-bmm-model-with-or-without-isotopic-ratio}}

Run BMM models for actual sediment samples (\emph{mix}) and virtual
mixtures (\emph{VM}). The BMM model performs a Bayesian un-mixing with a
Monte-Carlo chain, the prediction is corrected using the sum of squared
relative error of each tracer. Without isotopic ratio within the
tracers, there is no need to take any precautions when setting up the
model.

\begin{Shaded}
\begin{Highlighting}[]
\CommentTok{\# Run BMM model for sediment samples}
\NormalTok{BMM.mix }\OtherTok{\textless{}{-}}\NormalTok{ fingR}\SpecialCharTok{::}\FunctionTok{run.BMM}\NormalTok{(}\AttributeTok{data =}\NormalTok{ database,                                           }\CommentTok{\# Dataset containing source and target samples}
                          \AttributeTok{class =} \StringTok{"Class\_decontamination"}\NormalTok{,                           }\CommentTok{\# Column containing the classification or grouping of sources and mixtures}
                          \AttributeTok{mixture =} \StringTok{"Target"}\NormalTok{,                                        }\CommentTok{\# Column name identifying the target samples}
                          \AttributeTok{sample.id =} \StringTok{"Sample\_name"}\NormalTok{,                                 }\CommentTok{\# Column name for sample identifiers}
                          \AttributeTok{tracers =}\NormalTok{ tracers}\SpecialCharTok{$}\NormalTok{msd\_KS,                                  }\CommentTok{\# Character vector containing tracers to consider}
                          \AttributeTok{uncertainty =} \FunctionTok{unname}\NormalTok{(prop.uncertainties[tracers}\SpecialCharTok{$}\NormalTok{msd\_KS]),  }\CommentTok{\# Optional: Character vector containing uncertainty of the tracers}
                          \AttributeTok{n.iter =} \DecValTok{30}\NormalTok{,                                               }\CommentTok{\# Number of iterations for the model (30 for test version {-} 2500 or 5000 iterations are recommended) \textquotesingle{}prop.uncertainties\textquotesingle{}}
                          \AttributeTok{save.dir =}\NormalTok{ dir.mod.BMM,                                    }\CommentTok{\# Optional: Directory path for saving the results {-} \textquotesingle{}BMM\_previsions.CSV\textquotesingle{}}
                          \CommentTok{\#note = "example"                                          \# Optional: Additional note to append to the file name}
\NormalTok{                          )}
\end{Highlighting}
\end{Shaded}

\begin{Shaded}
\begin{Highlighting}[]
\CommentTok{\# Run BMM model for virtual mixtures}
\NormalTok{BMM.VM }\OtherTok{\textless{}{-}}\NormalTok{ fingR}\SpecialCharTok{::}\FunctionTok{run.BMM}\NormalTok{(}\AttributeTok{data =}\NormalTok{ VM}\SpecialCharTok{$}\NormalTok{full,                                            }\CommentTok{\# Dataset containing source and target samples}
                         \AttributeTok{class =} \StringTok{"Class\_decontamination"}\NormalTok{,                           }\CommentTok{\# Column containing the classification or grouping of sources and mixtures}
                         \AttributeTok{mixture =} \StringTok{"Virtual Mixture"}\NormalTok{,                               }\CommentTok{\# Column name identifying the target samples}
                         \AttributeTok{sample.id =} \StringTok{"Sample\_name"}\NormalTok{,                                 }\CommentTok{\# Column name for sample identifiers}
                         \AttributeTok{tracers =}\NormalTok{ tracers}\SpecialCharTok{$}\NormalTok{msd\_KS,                                  }\CommentTok{\# Character vector containing tracers to consider}
                         \AttributeTok{uncertainty =} \FunctionTok{unname}\NormalTok{(prop.uncertainties[tracers}\SpecialCharTok{$}\NormalTok{msd\_KS]),  }\CommentTok{\# Optional: Character vector containing uncertainty of the tracers}
                         \AttributeTok{n.iter =} \DecValTok{30}\NormalTok{,                                               }\CommentTok{\# Number of iterations for the model (30 for test version {-} 2500 or 5000 iterations are recommended)}
                         \AttributeTok{save.dir =}\NormalTok{ dir.mod.BMM,                                    }\CommentTok{\# Optional: Directory path for saving the results {-} \textquotesingle{}BMM\_previsions\_VM.CSV\textquotesingle{}}
                         \AttributeTok{note =} \StringTok{"VM"}                                                \CommentTok{\# Optional: Additional note to append to the file name}
\NormalTok{                         )}
\end{Highlighting}
\end{Shaded}

When dealing with isotopic ratios, which are non-linear properties,
errors should be calculated considering relative property content (see
\href{https://doi.org/10.1002/hyp.10311}{Laceby et al.~(2015)} for
further details). For example, the delta 13C ratio indicates the
isotopic ratio of 12C to 13C in organic matter, the \texttt{run.BMM}
function should be configured in this way:

\begin{Shaded}
\begin{Highlighting}[]
\CommentTok{\# Run BMM model for sediment samples}
\NormalTok{BMM.iso }\OtherTok{\textless{}{-}}\NormalTok{ fingR}\SpecialCharTok{::}\FunctionTok{run.BMM}\NormalTok{(}\AttributeTok{data =}\NormalTok{ database,                                           }\CommentTok{\# Dataset containing source and target samples}
                             \AttributeTok{class =} \StringTok{"Class\_decontamination"}\NormalTok{,                           }\CommentTok{\# Column containing the classification or grouping of sources and mixtures}
                             \AttributeTok{mixture =} \StringTok{"Target"}\NormalTok{,                                        }\CommentTok{\# Column name identifying the target samples}
                             \AttributeTok{sample.id =} \StringTok{"Sample\_name"}\NormalTok{,                                 }\CommentTok{\# Column name for sample identifiers}
                             \AttributeTok{tracers =}\NormalTok{ tracers}\SpecialCharTok{$}\NormalTok{msd\_KS,                                  }\CommentTok{\# Character vector containing tracers to consider}
                             \AttributeTok{uncertainty =} \FunctionTok{unname}\NormalTok{(prop.uncertainties[tracers}\SpecialCharTok{$}\NormalTok{msd\_KS]),  }\CommentTok{\# Optional: Character vector containing uncertainty of the tracers}
                             \AttributeTok{isotope.ratio =} \FunctionTok{c}\NormalTok{(}\StringTok{"d13C\_PrM"}\NormalTok{),                             }\CommentTok{\# Optional: Character vector containing isotopic ratios}
                             \AttributeTok{isotope.prop =} \FunctionTok{c}\NormalTok{(}\StringTok{"TOC\_PrC"}\NormalTok{),                               }\CommentTok{\# Optional: Character vector containing isotopic ratios respective properties}
                             \AttributeTok{isotopes.unc =} \FunctionTok{c}\NormalTok{(}\StringTok{"d13C\_SD"}\NormalTok{),                               }\CommentTok{\# Optional: Character vecotr containing uncertainty of the isotopic ratios}
                             \AttributeTok{n.iter =} \DecValTok{30}\NormalTok{,                                               }\CommentTok{\# Number of iterations for the model (30 for test version {-} 2500 or 5000 iterations are recommended) \textquotesingle{}prop.uncertainties\textquotesingle{}}
                             \AttributeTok{save.dir =}\NormalTok{ dir.mod.BMM,                                    }\CommentTok{\# Optional: Directory path for saving the results {-} \textquotesingle{}BMM\_previsions.CSV\textquotesingle{}}
                             \CommentTok{\#note = "example"                                          \# Optional: Additional note to append to the file name}
\NormalTok{                             )}
\end{Highlighting}
\end{Shaded}

After running the models, we extract the prediction information from the
iteration previsions. The \texttt{BMM.summary} function provides a
summary of the predictions, including the mean, standard deviation, and
various quantiles (2.5, 5, 25, 50, 75, 95, 97.5\%) for each mixture
(sediment sample or virtual mixture). From this summary, the
\texttt{BMM.pred} function extracts the `Median' and/or `Mean' for each
mixture. Finally, the \texttt{ensure.total} function ensures that the
total predicted contribution from all sources sums to 1 or 100\%.

\begin{Shaded}
\begin{Highlighting}[]
\CommentTok{\# For sediment samples}
\DocumentationTok{\#\# Summarise BMM model previsions}
\NormalTok{BMM.summary.mix }\OtherTok{\textless{}{-}}\NormalTok{ fingR}\SpecialCharTok{::}\FunctionTok{BMM.summary}\NormalTok{(}\AttributeTok{pred =}\NormalTok{ BMM.mix,            }\CommentTok{\# Predicted contributions from BMM}
                                      \CommentTok{\#sample.id = "mix.names",  \# Column name for sample identifier}
                                      \CommentTok{\#source = "source",        \# Column name for source identifier}
                                      \CommentTok{\#value = "value",          \# Column name for prediction value identifier}
                                      \AttributeTok{save.dir =}\NormalTok{ dir.mod.BMM,    }\CommentTok{\# Optional: Directory path for saving the results}
                                      \CommentTok{\#note = "example"          \# Optional: Additional note to append to the file name}
\NormalTok{                                      )}

\DocumentationTok{\#\# Extracts the median value of the previsions}
\NormalTok{BMM.preds.mix }\OtherTok{\textless{}{-}}\NormalTok{ fingR}\SpecialCharTok{::}\FunctionTok{BMM.pred}\NormalTok{(}\AttributeTok{data =}\NormalTok{ BMM.summary.mix,         }\CommentTok{\# Summary statistics of the predicted contribution by BMM, data from fingR::BMM.summary.mix}
                                 \AttributeTok{stats =} \StringTok{"Median"}\NormalTok{,               }\CommentTok{\# The summary statistics for source contribution, Could be Mean or Median}
                                 \CommentTok{\#sample.id = "mix.names",       \# Column name for sample identifier}
                                 \CommentTok{\#source = "source",             \# Column name for prediction value identifier}
                                 \AttributeTok{save.dir =}\NormalTok{ dir.mod.BMM,         }\CommentTok{\# Optional: Directory path for saving the results}
                                 \CommentTok{\#note = "example"               \# Optional: Additional note to append to the file name}
\NormalTok{                                 )}

\DocumentationTok{\#\# Ensure that the total predicted contribution sums to 1 or 100\%}
\NormalTok{BMM.preds.mixE }\OtherTok{\textless{}{-}}\NormalTok{ fingR}\SpecialCharTok{::}\FunctionTok{ensure.total}\NormalTok{(}\AttributeTok{data =}\NormalTok{ BMM.preds.mix,      }\CommentTok{\# Predicted source contribution for each sample, data from fingR::BMM.pre}
                                      \AttributeTok{sample.name =} \StringTok{"mix.names"}\NormalTok{, }\CommentTok{\# Column name for sample identifier}
                                      \AttributeTok{path =}\NormalTok{ dir.mod.BMM,        }\CommentTok{\# Optional: Directory path for saving the results}
                                      \CommentTok{\#note = "example"          \# Optional: Additional note to append to the file name}
\NormalTok{                                      )}
\end{Highlighting}
\end{Shaded}

\begin{Shaded}
\begin{Highlighting}[]
\NormalTok{BMM.preds.mixE[}\DecValTok{1}\SpecialCharTok{:}\DecValTok{5}\NormalTok{,]}
\CommentTok{\#\textgreater{}           mix.names Median\_Forest Median\_Subsoil Median\_Undecontaminated total}
\CommentTok{\#\textgreater{} 1 ManoDd\_2106\_00{-}01         0.001          0.807                   0.192     1}
\CommentTok{\#\textgreater{} 2 ManoDd\_2106\_01{-}02         0.001          0.974                   0.025     1}
\CommentTok{\#\textgreater{} 3 ManoDd\_2106\_02{-}03         0.031          0.755                   0.214     1}
\CommentTok{\#\textgreater{} 4 ManoDd\_2106\_03{-}04         0.020          0.728                   0.252     1}
\CommentTok{\#\textgreater{} 5 ManoDd\_2106\_04{-}05         0.010          0.820                   0.170     1}
\end{Highlighting}
\end{Shaded}

Same code for virtual mixtures:

\begin{Shaded}
\begin{Highlighting}[]
\CommentTok{\# For virtual mixtures}
\DocumentationTok{\#\# Summarise BMM model previsions}
\NormalTok{BMM.summary.VM }\OtherTok{\textless{}{-}}\NormalTok{ fingR}\SpecialCharTok{::}\FunctionTok{BMM.summary}\NormalTok{(}\AttributeTok{pred =}\NormalTok{ BMM.VM,              }\CommentTok{\# Predicted contributions from BMM}
                                     \CommentTok{\#sample.id = "mix.names",   \# Column name for sample identifier}
                                     \CommentTok{\#source = "source",         \# Column name for source identifier}
                                     \CommentTok{\#value = "value",           \# Column name for prediction value identifier}
                                     \AttributeTok{save.dir =}\NormalTok{ dir.mod.BMM,     }\CommentTok{\# Optional: Directory path for saving the results}
                                     \AttributeTok{note =} \StringTok{"VM"}                 \CommentTok{\# Optional: Additional note to append to the file name}
\NormalTok{                                     )}

\DocumentationTok{\#\# Extracts the median value of the previsions}
\NormalTok{BMM.preds.VM }\OtherTok{\textless{}{-}}\NormalTok{ fingR}\SpecialCharTok{::}\FunctionTok{BMM.pred}\NormalTok{(}\AttributeTok{data =}\NormalTok{ BMM.summary.VM,           }\CommentTok{\# Summary statistics of the predicted contribution by BMM, data from fingR::BMM.summary.mix}
                                \AttributeTok{stats =} \StringTok{"Median"}\NormalTok{,                }\CommentTok{\# The summary statistics for source contribution, Could be Mean or Median}
                                \CommentTok{\#sample.id = "mix.names",        \# Column name for sample identifier}
                                \CommentTok{\#source = "source",              \# Column name for prediction value identifier}
                                \AttributeTok{save.dir =}\NormalTok{ dir.mod.BMM,          }\CommentTok{\# Optional: Directory path for saving the results}
                                \AttributeTok{note =} \StringTok{"VM"}                      \CommentTok{\# Optional: Additional note to append to the file name}
\NormalTok{                                 )}

\DocumentationTok{\#\# Ensure that the total predicted contribution sums to 1 or 100\%}
\NormalTok{BMM.preds.VME }\OtherTok{\textless{}{-}}\NormalTok{ fingR}\SpecialCharTok{::}\FunctionTok{ensure.total}\NormalTok{(}\AttributeTok{data =}\NormalTok{ BMM.preds.VM,        }\CommentTok{\# Predicted source contribution for each sample, data from fingR::BMM.pre}
                                      \AttributeTok{sample.name =} \StringTok{"mix.names"}\NormalTok{, }\CommentTok{\# Column name for sample identifier}
                                      \AttributeTok{path =}\NormalTok{ dir.mod.BMM,        }\CommentTok{\# Optional: Directory path for saving the results}
                                      \AttributeTok{note =} \StringTok{"VM"}                \CommentTok{\# Optional: Additional note to append to the file name}
\NormalTok{                                      )}
\end{Highlighting}
\end{Shaded}

\begin{Shaded}
\begin{Highlighting}[]
\NormalTok{BMM.preds.VME[}\DecValTok{1}\SpecialCharTok{:}\DecValTok{5}\NormalTok{,]}
\CommentTok{\#\textgreater{}   mix.names Median\_Forest Median\_Subsoil Median\_Undecontaminated total}
\CommentTok{\#\textgreater{} 1    VM{-}001         0.001          0.677                   0.322     1}
\CommentTok{\#\textgreater{} 2    VM{-}002         0.035          0.763                   0.202     1}
\CommentTok{\#\textgreater{} 3    VM{-}003         0.030          0.969                   0.001     1}
\CommentTok{\#\textgreater{} 4    VM{-}004         0.174          0.825                   0.001     1}
\CommentTok{\#\textgreater{} 5    VM{-}005         0.001          0.896                   0.103     1}
\end{Highlighting}
\end{Shaded}

\hypertarget{modelling-accuracy-statistics}{%
\paragraph{Modelling accuracy
statistics}\label{modelling-accuracy-statistics}}

The modelling accuracy of BMM model is evaluate with the virtual
mixtures. These virtual mixtures, serving as target samples with known
contributions (\emph{VM.contrib}), allow for the calculation of
modelling accuracy metrics based on their prediction.

The \texttt{eval.groups} function calculates several common modelling
accuracy metrics: ME, RMSE, squared Pearson's correlation coefficient
(r2), and Nash-Sutcliffe Modelling Efficiency Coefficient (NSE).

\begin{Shaded}
\begin{Highlighting}[]
\NormalTok{BMM.stats }\OtherTok{\textless{}{-}}\NormalTok{ fingR}\SpecialCharTok{::}\FunctionTok{eval.groups}\NormalTok{(}\AttributeTok{df.obs =}\NormalTok{ VM.contrib,                               }\CommentTok{\# Theoretical contribution}
                                \AttributeTok{df.pred =}\NormalTok{ BMM.preds.VME }\SpecialCharTok{\%\textgreater{}\%}\NormalTok{ dplyr}\SpecialCharTok{::}\FunctionTok{select}\NormalTok{(}\SpecialCharTok{{-}}\NormalTok{total), }\CommentTok{\# Predicted contribution (remove the $total column from ensured data.frame)}
                                \AttributeTok{by =} \FunctionTok{c}\NormalTok{(}\StringTok{"Sample\_name"} \OtherTok{=} \StringTok{"mix.names"}\NormalTok{),               }\CommentTok{\# Column where mixtures labels are specified (for \textasciigrave{}dplyr::left\_join\textasciigrave{} function)}
                                \AttributeTok{path =}\NormalTok{ dir.mod.BMM,                                }\CommentTok{\# Optional: Directory path for saving the results}
                                \CommentTok{\#note = "example"                                  \# Optional: Additional note to append to the file name}
\NormalTok{                                )}
\end{Highlighting}
\end{Shaded}

\begin{Shaded}
\begin{Highlighting}[]
\NormalTok{BMM.stats}
\CommentTok{\#\textgreater{}     Type           Source    ME RMSE   r2   NSE}
\CommentTok{\#\textgreater{} 1 Median           Forest {-}0.17 0.24 0.57  0.10}
\CommentTok{\#\textgreater{} 2 Median          Subsoil  0.41 0.48 0.38 {-}2.58}
\CommentTok{\#\textgreater{} 3 Median Undecontaminated {-}0.24 0.37 0.02 {-}1.18}
\end{Highlighting}
\end{Shaded}

The \texttt{CRPS} functions calculate the continuous ranking probability
score and returns a list contraining two \emph{data.frame} objects; one
with the \texttt{\$samples} CRPS values per source class group (saved as
\emph{CRPS.csv}), the other is \texttt{\$mean} with the mean of the CRPS
per source class groups (saved as \emph{CRPS\_mean.csv}).

\begin{Shaded}
\begin{Highlighting}[]
\CommentTok{\# Calculate prediction CRPS values}
\NormalTok{BMM.CRPS }\OtherTok{\textless{}{-}}\NormalTok{ fingR}\SpecialCharTok{::}\FunctionTok{CRPS}\NormalTok{(}\AttributeTok{obs =}\NormalTok{ VM.contrib,                                             }\CommentTok{\# Observed contributions}
                        \AttributeTok{prev =} \FunctionTok{read.csv}\NormalTok{(}\FunctionTok{paste0}\NormalTok{(dir.mod.BMM, }\StringTok{"BMM\_prevision\_VM.csv"}\NormalTok{)), }\CommentTok{\# Predicted prevision from BMM saved by \textasciigrave{}rum.BMM()\textasciigrave{}}
                        \AttributeTok{source.groups =} \FunctionTok{c}\NormalTok{(}\StringTok{"Forest"}\NormalTok{, }\StringTok{"Subsoil"}\NormalTok{, }\StringTok{"Undecontaminated"}\NormalTok{),   }\CommentTok{\# Source class groups}
                        \AttributeTok{mean.cal =} \ConstantTok{TRUE}\NormalTok{,                                              }\CommentTok{\# Calculate mean CRPS per source class group}
                        \AttributeTok{save.dir =}\NormalTok{ dir.mod.BMM,                                       }\CommentTok{\# Optional: Directory path for saving the results}
                        \CommentTok{\#note = "example"                                             \# Optional: Additional note to append to the file name}
\NormalTok{                        )}
\CommentTok{\#\textgreater{} Le chargement a nécessité le package : scoringRules}
\end{Highlighting}
\end{Shaded}

\begin{Shaded}
\begin{Highlighting}[]
\NormalTok{BMM.CRPS}\SpecialCharTok{$}\NormalTok{samples[}\DecValTok{1}\SpecialCharTok{:}\DecValTok{6}\NormalTok{,]}
\CommentTok{\#\textgreater{}   Sample\_name Forest Subsoil Undecontaminated}
\CommentTok{\#\textgreater{} 1      VM{-}001 0.0306  0.3142           0.4747}
\CommentTok{\#\textgreater{} 2      VM{-}002 0.0499  0.2716           0.4404}
\CommentTok{\#\textgreater{} 3      VM{-}003 0.0507  0.3733           0.5509}
\CommentTok{\#\textgreater{} 4      VM{-}004 0.0884  0.2612           0.5035}
\CommentTok{\#\textgreater{} 5      VM{-}005 0.0320  0.2264           0.3282}
\CommentTok{\#\textgreater{} 6      VM{-}006 0.0694  0.2602           0.5294}

\NormalTok{BMM.CRPS}\SpecialCharTok{$}\NormalTok{mean}
\CommentTok{\#\textgreater{}             Source CRPS.mean}
\CommentTok{\#\textgreater{} 1           Forest    0.1389}
\CommentTok{\#\textgreater{} 2          Subsoil    0.1651}
\CommentTok{\#\textgreater{} 3 Undecontaminated    0.1818}
\end{Highlighting}
\end{Shaded}

The \texttt{interval.width} functions calculate two prediction interval
width: The \emph{W50} contains 50\% of the prevision (Q75-Q25) and the
\emph{W95} contains 95\% of the prevision (Q97.5-Q2.5). It returns a
list contraining two \emph{data.frame} objects; one with the
\texttt{\$samples} prediction interval width values per source class
group (saved as \emph{Interval\_width.csv}), the other is
\texttt{\$mean} with the mean of the prediction interval width per
source class groups (saved as \emph{Interval\_width\_mean.csv}).

\begin{Shaded}
\begin{Highlighting}[]
\CommentTok{\# Calculate prediction interval width (W95, W50)}
\NormalTok{BMM.predWidth }\OtherTok{\textless{}{-}}\NormalTok{ fingR}\SpecialCharTok{::}\FunctionTok{interval.width}\NormalTok{(}\AttributeTok{path.to.prev =} \FunctionTok{paste0}\NormalTok{(dir.mod.BMM, }\StringTok{"BMM\_prevision\_VM.csv"}\NormalTok{), }\CommentTok{\# Path to prediction file}
                                       \AttributeTok{mean.cal =} \ConstantTok{TRUE}\NormalTok{,                                            }\CommentTok{\# Calculate mean of interval width per source group}
                                       \AttributeTok{save =} \ConstantTok{TRUE}\NormalTok{,                                                }\CommentTok{\# Save the results at the same location of the path.to.prev}
                                       \CommentTok{\#note = "exemple"                                           \# Optional: Additional note to append to the file name}
\NormalTok{                                       )}
\end{Highlighting}
\end{Shaded}

\begin{Shaded}
\begin{Highlighting}[]
\NormalTok{BMM.predWidth}\SpecialCharTok{$}\NormalTok{samples[}\DecValTok{1}\SpecialCharTok{:}\DecValTok{6}\NormalTok{,]}
\CommentTok{\#\textgreater{}   mix.names           source   W50   W95}
\CommentTok{\#\textgreater{} 1    VM{-}001           Forest 0.154 0.796}
\CommentTok{\#\textgreater{} 2    VM{-}001          Subsoil 0.728 0.997}
\CommentTok{\#\textgreater{} 3    VM{-}001 Undecontaminated 0.614 0.990}
\CommentTok{\#\textgreater{} 4    VM{-}002           Forest 0.297 0.768}
\CommentTok{\#\textgreater{} 5    VM{-}002          Subsoil 0.623 0.997}
\CommentTok{\#\textgreater{} 6    VM{-}002 Undecontaminated 0.652 0.942}

\NormalTok{BMM.predWidth}\SpecialCharTok{$}\NormalTok{mean}
\CommentTok{\#\textgreater{} \# A tibble: 3 x 3}
\CommentTok{\#\textgreater{}   Source           W50.mean W95.mean}
\CommentTok{\#\textgreater{}   \textless{}chr\textgreater{}               \textless{}dbl\textgreater{}    \textless{}dbl\textgreater{}}
\CommentTok{\#\textgreater{} 1 Forest              0.356    0.883}
\CommentTok{\#\textgreater{} 2 Subsoil             0.59     0.969}
\CommentTok{\#\textgreater{} 3 Undecontaminated    0.408    0.914}
\end{Highlighting}
\end{Shaded}

The \texttt{ESP} function calculates the Encompassed Sample Prediction
(ESP). The ESP is a newly introduced statistics in
\href{}{Chalaux-Clergue et al (under review)} and was created to assess
the transferability of the statistics calculated on virtual mixtures to
actual sediment samples. The ESP was calculated as the percentage of
actual samples for which the predicted contributions remained within the
lowest and the highest predicted contributions obtained for the virtual
mixtures. When expressed as a percentage, ESP ranges from 0 to 100\%,
the latter providing an optimal value. Values close to 100\% indicate a
higher transferability of modelling evaluation statistics calculated on
virtual mixture to actual sediment samples.

\begin{Shaded}
\begin{Highlighting}[]
\NormalTok{sources.lvl }\OtherTok{\textless{}{-}} \FunctionTok{c}\NormalTok{(}\StringTok{"Forest"}\NormalTok{, }\StringTok{"Subsoil"}\NormalTok{, }\StringTok{"Undecontaminated"}\NormalTok{)}

\CommentTok{\# Calculate encompassed sample predictions (ESP)}
\NormalTok{BMM.ESP }\OtherTok{\textless{}{-}}\NormalTok{ fingR}\SpecialCharTok{::}\FunctionTok{ESP}\NormalTok{(}\AttributeTok{obs =}\NormalTok{ BMM.preds.VM,                       }\CommentTok{\# Virtual mixtures predicted contributions}
                      \AttributeTok{pred =}\NormalTok{ BMM.preds.mixE,                    }\CommentTok{\# Actual sediment samples predicted contributions}
                      \AttributeTok{sources =} \FunctionTok{paste0}\NormalTok{(}\StringTok{"Median\_"}\NormalTok{, sources.lvl), }\CommentTok{\# Sources labels in prediction objects}
                      \AttributeTok{count =} \StringTok{"Both"}                            \CommentTok{\# Count \textquotesingle{}Number\textquotesingle{} and \textquotesingle{}Percentage\textquotesingle{}}
\NormalTok{                      )}
\end{Highlighting}
\end{Shaded}

\begin{Shaded}
\begin{Highlighting}[]
\NormalTok{BMM.ESP}
\CommentTok{\#\textgreater{}                                   Source ESP.Number ESP.Percentage}
\CommentTok{\#\textgreater{} Median\_Forest                     Forest         32             84}
\CommentTok{\#\textgreater{} Median\_Subsoil                   Subsoil         37             97}
\CommentTok{\#\textgreater{} Median\_Undecontaminated Undecontaminated         26             68}
\end{Highlighting}
\end{Shaded}

Modelling accuracy statistics could be interpreted the following way:
``Higher values of W50 indicate a wider distribution, which is related
to a higher uncertainty. The sign of the ME indicates the direction of
the bias, i.e.~an overestimation or underestimation (positive or
negative value, respectively). As ME is affected by cancellation, a ME
of zero can also reflect a balanced distribution of predictions around
the 1 : 1 line. Although this is not a bias, it does not mean that the
model outputs are devoid of errors. The RMSE is a measure of the
accuracy and allows us to calculate prediction errors of different
models for a particular dataset. RMSE is always positive, and its ideal
value is zero, which indicates a perfect fit to the data. As RMSE
depends on the squared error, it is sensitive to outliers. The r2
describes how linear the prediction is. The NSE indicates the magnitude
of variance explained by the model, i.e.~how well the predictions match
with the observations. A negative RMSE indicates that the mean of the
measured values provides a better predictor than the model. The joint
use of r2 and NSE allows for a better appreciation of the distribution
shape of predictions and thus facilitates the understanding of the
nature of model prediction errors. The CRPS evaluates both the accuracy
and sharpness (i.e.~precision) of a distribution of predicted continuous
values from a probabilistic model for each sample (Matheson and Winkler,
1976). The CRPS is minimised when the observed value corresponds to a
high probability value in the distribution of model outputs.''
\href{10.5194/soil-10-109-2024}{(Chalaux-Clergue et al, 2024)}.

\hypertarget{mixsiar}{%
\subsubsection{MixSIAR}\label{mixsiar}}

The \texttt{MixSIAR} is an R package designed to create and run Bayesian
mixing models. This package is widely used in the sediment source
fingerprinting community to predict source contribution. To explore more
about \texttt{MixSIAR}, including detailed tutorials, examples, and
technical documentation, please visit the official
\href{http://brianstock.github.io/MixSIAR/index.html}{MixSIAR website}.
Additionally, the source code and further resources can be found on the
\href{https://github.com/brianstock/MixSIAR}{MixSIAR GitHub page}.

According to
\href{http://brianstock.github.io/MixSIAR/index.html\#installation}{MixSIAR
guide}, installation should follow these steps:

\begin{enumerate}
\def\labelenumi{\arabic{enumi}.}
\tightlist
\item
  Download and install/updata \href{https://cran.r-project.org/}{R}.
\item
  Download and install \href{http://mcmc-jags.sourceforge.net/}{JAGS}.
\item
  Open R and run:
\end{enumerate}

\begin{Shaded}
\begin{Highlighting}[]
\FunctionTok{install.packages}\NormalTok{(}\StringTok{"MixSIAR"}\NormalTok{, }\AttributeTok{dependencies=}\ConstantTok{TRUE}\NormalTok{)}
\end{Highlighting}
\end{Shaded}

You can install the GitHub version

\begin{Shaded}
\begin{Highlighting}[]
\CommentTok{\#install.packages(remotes)}

\NormalTok{remotes}\SpecialCharTok{::}\FunctionTok{install\_github}\NormalTok{(}\StringTok{"brianstock/MixSIAR"}\NormalTok{, }\AttributeTok{dependencies=}\NormalTok{T)}
\end{Highlighting}
\end{Shaded}

Create a folder specific from BMM modelling results.

\begin{Shaded}
\begin{Highlighting}[]
\CommentTok{\# Create new folder to save BMM modelling results}
\FunctionTok{dir.create}\NormalTok{(}\FunctionTok{file.path}\NormalTok{(dir.modelling, }\StringTok{"MixSIAR/"}\NormalTok{), }\AttributeTok{showWarnings =} \ConstantTok{FALSE}\NormalTok{)}
\NormalTok{dir.mod.MixSIAR }\OtherTok{\textless{}{-}} \FunctionTok{paste0}\NormalTok{(dir.modelling, }\StringTok{"MixSIAR/"}\NormalTok{)}
\end{Highlighting}
\end{Shaded}

\hypertarget{generate-data-for-mixsiar}{%
\paragraph{Generate data for MixSIAR}\label{generate-data-for-mixsiar}}

To MixSIAR models require data in a specific format to load the
information of mixtures and sources samples. The
\texttt{data.for.MixSIAR} function generates \emph{csv} files that
conform to the format required by MixSIAR loading functions. The
function generates three files: \emph{MixSIAR\_mix.csv} containing
mixtures information, \emph{MixSIAR\_sources} containing the mean and
standard deviation (sd) of the source classes, and
\emph{MixSIAR\_discrimination} which is a matrix of zero as there is no
throphic information in sediment source fingerprinting studies.

Of note, if several selection of tracers were obtained from the tracer
selection different files should be created. Use the \texttt{note}
argument to differentiate them.

\begin{Shaded}
\begin{Highlighting}[]
\NormalTok{fingR}\SpecialCharTok{::}\FunctionTok{data.for.MixSIAR}\NormalTok{(}\AttributeTok{data =}\NormalTok{ database,                 }\CommentTok{\# Dataset containing source samples}
                        \AttributeTok{class =} \StringTok{"Class\_decontamination"}\NormalTok{, }\CommentTok{\# Column containing the classification or grouping of sources and mixtures}
                        \AttributeTok{target =} \StringTok{"Target"}\NormalTok{,               }\CommentTok{\# Identifier for mixture samples within the class column}
                        \AttributeTok{tracers =}\NormalTok{ tracers}\SpecialCharTok{$}\NormalTok{msd\_KS,        }\CommentTok{\# Character vector containing tracers to consider}
                        \AttributeTok{sample.name =} \StringTok{"Sample\_name"}\NormalTok{,     }\CommentTok{\# Column containing sample names in data}
                        \AttributeTok{save.dir =}\NormalTok{ dir.mod.MixSIAR,      }\CommentTok{\# Directory path for saving the files}
                        \CommentTok{\# note = "exemple",              \# Optional: Additional note to append to the file name}
                        \CommentTok{\# fileEncoding = "latin1",       \# Optional: File encoding, important if special character are used in source levels}
                        \CommentTok{\# show.data = FALSE,             \# Optional: Return generated files in R}
\NormalTok{                        )}
\end{Highlighting}
\end{Shaded}

\begin{Shaded}
\begin{Highlighting}[]
\NormalTok{fingR}\SpecialCharTok{::}\FunctionTok{data.for.MixSIAR}\NormalTok{(}\AttributeTok{data =}\NormalTok{ VM}\SpecialCharTok{$}\NormalTok{full,                  }\CommentTok{\# Dataset containing source samples}
                        \AttributeTok{class =} \StringTok{"Class\_decontamination"}\NormalTok{, }\CommentTok{\# Column containing the classification or grouping of sources and mixtures}
                        \AttributeTok{target =} \StringTok{"Virtual Mixture"}\NormalTok{,      }\CommentTok{\# Identifier for mixture samples within the class column}
                        \AttributeTok{tracers =}\NormalTok{ tracers}\SpecialCharTok{$}\NormalTok{msd\_KS,        }\CommentTok{\# Character vector containing tracers to consider}
                        \AttributeTok{sample.name =} \StringTok{"Sample\_name"}\NormalTok{,     }\CommentTok{\# Column containing sample names in data}
                        \AttributeTok{save.dir =}\NormalTok{ dir.mod.MixSIAR,      }\CommentTok{\# Directory path for saving the files}
                        \AttributeTok{note =} \StringTok{"VM"}\NormalTok{,                     }\CommentTok{\# Optional: Additional note to append to the file name}
                        \CommentTok{\# fileEncoding = "latin1",       \# Optional: File encoding, important if special character are used in source levels}
                        \CommentTok{\# show.data = FALSE,             \# Optional: Return generated files in R}
\NormalTok{                        )}
\end{Highlighting}
\end{Shaded}

\hypertarget{load-mixture-source-and-discrimination-data}{%
\paragraph{Load mixture, source and discrimination
data}\label{load-mixture-source-and-discrimination-data}}

Load mixture, source and discrimination data for sediment samples.

\begin{Shaded}
\begin{Highlighting}[]
\FunctionTok{library}\NormalTok{(MixSIAR)}

\CommentTok{\# Load sediment samples data}
\NormalTok{MSIAR.mix  }\OtherTok{\textless{}{-}}\NormalTok{ MixSIAR}\SpecialCharTok{::}\FunctionTok{load\_mix\_data}\NormalTok{(}\AttributeTok{filename =} \FunctionTok{paste0}\NormalTok{(dir.mod.MixSIAR, }\StringTok{"MixSIAR\_mix.csv"}\NormalTok{),               }\CommentTok{\# File containing real samples data}
                                     \AttributeTok{iso\_names =}\NormalTok{ tracers}\SpecialCharTok{$}\NormalTok{msd\_KS,                                          }\CommentTok{\# Names of tracers}
                                     \AttributeTok{factors =} \FunctionTok{c}\NormalTok{(}\StringTok{"Sample\_name"}\NormalTok{),                                          }\CommentTok{\# Columns used to differentiate samples}
                                     \AttributeTok{fac\_random =} \ConstantTok{FALSE}\NormalTok{,                                                  }\CommentTok{\# Indicates if the factor is a random effect}
                                     \AttributeTok{cont\_effects =} \ConstantTok{NULL}                                                  \CommentTok{\# Continuous effect column not specified}
\NormalTok{                                     )}

\CommentTok{\# Load source data}
\NormalTok{MSIAR.source }\OtherTok{\textless{}{-}}\NormalTok{ MixSIAR}\SpecialCharTok{::}\FunctionTok{load\_source\_data}\NormalTok{(}\AttributeTok{filename =} \FunctionTok{paste0}\NormalTok{(dir.mod.MixSIAR, }\StringTok{"MixSIAR\_sources.csv"}\NormalTok{),      }\CommentTok{\# File containing source data}
                                          \AttributeTok{source\_factors =} \ConstantTok{NULL}\NormalTok{,                                          }\CommentTok{\# No source factors specified}
                                          \AttributeTok{conc\_dep =} \ConstantTok{FALSE}\NormalTok{,                                               }\CommentTok{\# Concentration dependence not considered}
                                          \AttributeTok{data\_type =} \StringTok{"means"}\NormalTok{,                                            }\CommentTok{\# Type of data provided is means}
                                          \AttributeTok{mix =}\NormalTok{ MSIAR.mix                                                 }\CommentTok{\# Actual samples mixtures}
\NormalTok{                                          )}

\CommentTok{\# Load discrimination data}
\NormalTok{MSIAR.discr }\OtherTok{\textless{}{-}}\NormalTok{ MixSIAR}\SpecialCharTok{::}\FunctionTok{load\_discr\_data}\NormalTok{(}\AttributeTok{filename =} \FunctionTok{paste0}\NormalTok{(dir.mod.MixSIAR, }\StringTok{"MixSIAR\_discrimination.csv"}\NormalTok{), }\CommentTok{\# File containing discrimination data}
                                        \AttributeTok{mix =}\NormalTok{ MSIAR.mix)                                                  }\CommentTok{\# Actual samples mixtures}
\end{Highlighting}
\end{Shaded}

Load mixture, source and discrimination data for virtual mixtures.

\begin{Shaded}
\begin{Highlighting}[]
\FunctionTok{library}\NormalTok{(MixSIAR)}


\CommentTok{\# Load virtual mixtures data}
\NormalTok{MSIAR.VM }\OtherTok{\textless{}{-}}\NormalTok{ MixSIAR}\SpecialCharTok{::}\FunctionTok{load\_mix\_data}\NormalTok{(}\AttributeTok{filename =} \FunctionTok{paste0}\NormalTok{(dir.mod.MixSIAR, }\StringTok{"MixSIAR\_mix\_VM.csv"}\NormalTok{),                    }\CommentTok{\# File containing virtual mixtures data}
                                          \AttributeTok{iso\_names =}\NormalTok{ tracers}\SpecialCharTok{$}\NormalTok{msd\_KS,                                           }\CommentTok{\# Names of tracers}
                                          \AttributeTok{factors =} \FunctionTok{c}\NormalTok{(}\StringTok{"Sample\_name"}\NormalTok{),                                           }\CommentTok{\# Columns used to differentiate samples}
                                          \AttributeTok{fac\_random =} \ConstantTok{FALSE}\NormalTok{,                                                   }\CommentTok{\# Indicates if the factor is a random effect}
                                          \AttributeTok{cont\_effects =} \ConstantTok{NULL}\NormalTok{)                                                  }\CommentTok{\# Continuous effect column not specified}


\CommentTok{\# Load source data}
\NormalTok{MSIAR.source.VM }\OtherTok{\textless{}{-}}\NormalTok{ MixSIAR}\SpecialCharTok{::}\FunctionTok{load\_source\_data}\NormalTok{(}\AttributeTok{filename =} \FunctionTok{paste0}\NormalTok{(dir.mod.MixSIAR, }\StringTok{"MixSIAR\_sources\_VM.csv"}\NormalTok{),      }\CommentTok{\# File containing source data}
                                          \AttributeTok{source\_factors =} \ConstantTok{NULL}\NormalTok{,                                                }\CommentTok{\# No source factors specified}
                                          \AttributeTok{conc\_dep =} \ConstantTok{FALSE}\NormalTok{,                                                     }\CommentTok{\# Concentration dependence not considered}
                                          \AttributeTok{data\_type =} \StringTok{"means"}\NormalTok{,                                                  }\CommentTok{\# Type of data provided is means}
                                          \AttributeTok{mix =}\NormalTok{ MSIAR.VM                                                        }\CommentTok{\# Actual samples mixtures}
\NormalTok{                                          )}

\CommentTok{\# Load discrimination data}
\NormalTok{MSIAR.discr.VM }\OtherTok{\textless{}{-}}\NormalTok{ MixSIAR}\SpecialCharTok{::}\FunctionTok{load\_discr\_data}\NormalTok{(}\AttributeTok{filename =} \FunctionTok{paste0}\NormalTok{(dir.mod.MixSIAR, }\StringTok{"MixSIAR\_discrimination\_VM.csv"}\NormalTok{), }\CommentTok{\# File containing discrimination data}
                                        \AttributeTok{mix =}\NormalTok{ MSIAR.VM)                                                         }\CommentTok{\# Actual samples mixtures}
\end{Highlighting}
\end{Shaded}

\hypertarget{write-jags-model-file}{%
\paragraph{Write JAGS model file}\label{write-jags-model-file}}

Write the JAGS file, which define model structure. The model will be
saved as \texttt{model\_file} (``MixSIAR\_model.txt'' is default).

\begin{Shaded}
\begin{Highlighting}[]
\CommentTok{\# Write JAGS model file for actual samples}
\NormalTok{MixSIAR}\SpecialCharTok{::}\FunctionTok{write\_JAGS\_model}\NormalTok{(}\AttributeTok{filename =} \FunctionTok{paste0}\NormalTok{(dir.mod.MixSIAR, }\StringTok{"MixSIAR\_model.txt"}\NormalTok{),    }\CommentTok{\# File path and name to write the JAGS model}
                          \AttributeTok{resid\_err =} \ConstantTok{FALSE}\NormalTok{,                                          }\CommentTok{\# Whether to include residual error in the model}
                          \AttributeTok{process\_err =} \ConstantTok{TRUE}\NormalTok{,                                         }\CommentTok{\# Whether to include process error in the model}
                          \AttributeTok{mix =}\NormalTok{ MSIAR.mix,                                            }\CommentTok{\# Actual samples mixtures dataset}
                          \AttributeTok{source =}\NormalTok{ MSIAR.source                                       }\CommentTok{\# Source dataset}
\NormalTok{                          )}
\end{Highlighting}
\end{Shaded}

\begin{Shaded}
\begin{Highlighting}[]
\CommentTok{\# Write JAGS model file for virtual mixtures}
\NormalTok{MixSIAR}\SpecialCharTok{::}\FunctionTok{write\_JAGS\_model}\NormalTok{(}\AttributeTok{filename =} \FunctionTok{paste0}\NormalTok{(dir.mod.MixSIAR, }\StringTok{"MixSIAR\_model\_VM.txt"}\NormalTok{), }\CommentTok{\# File path and name to write the JAGS model}
                          \AttributeTok{resid\_err =} \ConstantTok{FALSE}\NormalTok{,                                          }\CommentTok{\# Whether to include residual error in the model}
                          \AttributeTok{process\_err =} \ConstantTok{TRUE}\NormalTok{,                                         }\CommentTok{\# Whether to include process error in the model}
                          \AttributeTok{mix =}\NormalTok{ MSIAR.VM,                                             }\CommentTok{\# Virtual mixtures dataset}
                          \AttributeTok{source =}\NormalTok{ MSIAR.source.VM                                    }\CommentTok{\# Source dataset loaded with virtual mixture mix}
\NormalTok{                          )}
\end{Highlighting}
\end{Shaded}

\hypertarget{run-mixsiar-model}{%
\paragraph{Run MixSIAR model}\label{run-mixsiar-model}}

When running MixSIAR model you should choose one of the
\href{http://brianstock.github.io/MixSIAR/articles/wolves_ex.html\#run-model}{MCMC
run option}. Here \texttt{run} is set to ``test'' as it is an example.

\begin{Shaded}
\begin{Highlighting}[]
\CommentTok{\# note if "Error: .onload ... \textquotesingle{}rgags\textquotesingle{} {-}\textgreater{} it\textquotesingle{}s because R version is too old need at least R.2.2}

\CommentTok{\# Run MixSIAR model for sediment samples}
\NormalTok{jags.mix }\OtherTok{\textless{}{-}}\NormalTok{ MixSIAR}\SpecialCharTok{::}\FunctionTok{run\_model}\NormalTok{(}\AttributeTok{run =} \StringTok{"test"}\NormalTok{,                                                 }\CommentTok{\# Type of run (e.g. "test", "long"...)}
                               \AttributeTok{mix =}\NormalTok{ MSIAR.mix,                                              }\CommentTok{\# Sediment samples dataset}
                               \AttributeTok{source =}\NormalTok{ MSIAR.source,                                        }\CommentTok{\# Source dataset}
                               \AttributeTok{discr =}\NormalTok{ MSIAR.discr,                                          }\CommentTok{\# Discrimination dataset}
                               \AttributeTok{model\_filename =} \FunctionTok{paste0}\NormalTok{(dir.mod.MixSIAR, }\StringTok{"MixSIAR\_model.txt"}\NormalTok{) }\CommentTok{\# File path to the JAGS model}
\NormalTok{                               )}
\CommentTok{\#\textgreater{} Compiling model graph}
\CommentTok{\#\textgreater{}    Resolving undeclared variables}
\CommentTok{\#\textgreater{}    Allocating nodes}
\CommentTok{\#\textgreater{} Graph information:}
\CommentTok{\#\textgreater{}    Observed stochastic nodes: 114}
\CommentTok{\#\textgreater{}    Unobserved stochastic nodes: 93}
\CommentTok{\#\textgreater{}    Total graph size: 2838}
\CommentTok{\#\textgreater{} }
\CommentTok{\#\textgreater{} Initializing model}
\end{Highlighting}
\end{Shaded}

\begin{Shaded}
\begin{Highlighting}[]
\CommentTok{\# note if "Error: .onload ... \textquotesingle{}rgags\textquotesingle{} {-}\textgreater{} it\textquotesingle{}s because R version is too old need at least R.2.2}

\CommentTok{\# Run MixSIAR model for Virtual mixtures}
\NormalTok{jags.VM }\OtherTok{\textless{}{-}}\NormalTok{ MixSIAR}\SpecialCharTok{::}\FunctionTok{run\_model}\NormalTok{(}\AttributeTok{run =} \StringTok{"test"}\NormalTok{,                                                    }\CommentTok{\# Type of run (e.g. "test", "long", "very long"...)}
                              \AttributeTok{mix =}\NormalTok{ MSIAR.VM,                                                  }\CommentTok{\# Virtual mixtures dataset}
                              \AttributeTok{source =}\NormalTok{ MSIAR.source.VM,                                        }\CommentTok{\# Source dataset loaded with virtual mixture mix}
                              \AttributeTok{discr =}\NormalTok{ MSIAR.discr.VM,                                          }\CommentTok{\# Discrimination dataset}
                              \AttributeTok{model\_filename =} \FunctionTok{paste0}\NormalTok{(dir.mod.MixSIAR, }\StringTok{"MixSIAR\_model\_VM.txt"}\NormalTok{) }\CommentTok{\# File path to the JAGS model}
\NormalTok{                              )}
\CommentTok{\#\textgreater{} Compiling model graph}
\CommentTok{\#\textgreater{}    Resolving undeclared variables}
\CommentTok{\#\textgreater{}    Allocating nodes}
\CommentTok{\#\textgreater{} Graph information:}
\CommentTok{\#\textgreater{}    Observed stochastic nodes: 693}
\CommentTok{\#\textgreater{}    Unobserved stochastic nodes: 479}
\CommentTok{\#\textgreater{}    Total graph size: 16541}
\CommentTok{\#\textgreater{} }
\CommentTok{\#\textgreater{} Initializing model}
\end{Highlighting}
\end{Shaded}

After running the models, we extract the prediction information from the
MixSIAR model predictions. The \texttt{MixSIAR.summary} function
provides a summary of the predictions, including the mean, standard
deviation, and various quantiles (2.5, 5, 25, 50, 75, 95, 97.5\%) for
each mixture (sediment sample or virtual mixture). From this summary,
the \texttt{MixSIAR.pred} function extracts the `Median' and/or `Mean'
for each mixture. Finally, the \texttt{ensure.total} function ensures
that the total predicted contribution from all sources sums to 1 or
100\%.

\begin{Shaded}
\begin{Highlighting}[]
\DocumentationTok{\#\# Summarise MixSIAR model previsions}
\NormalTok{MixSIAR.summary.mix }\OtherTok{\textless{}{-}}\NormalTok{ fingR}\SpecialCharTok{::}\FunctionTok{JAGS.summary}\NormalTok{(}\AttributeTok{jags.1 =}\NormalTok{ jags.mix,                        }\CommentTok{\# Data from the MixSIAR model \textasciigrave{}MixSIAR::run\_model()\textasciigrave{}}
                                          \AttributeTok{mix =}\NormalTok{ MSIAR.mix,                           }\CommentTok{\# Sediment dataset }
                                          \AttributeTok{sources =}\NormalTok{ MSIAR.source,                    }\CommentTok{\# Source dataset}
                                          \AttributeTok{path =}\NormalTok{ dir.mod.MixSIAR,                    }\CommentTok{\# Directory path for saving the files}
                                          \CommentTok{\#note = "example",                         \# Optional: Additional note to append to the file name}
                                          \AttributeTok{save\_pred =} \ConstantTok{TRUE}                           \CommentTok{\# Optional: Save the MixSIAR modelling predictions (heavy files)}
\NormalTok{                                          )}

\DocumentationTok{\#\# Extracts the median value of the previsions}
\NormalTok{MixSIAR.preds.mix }\OtherTok{\textless{}{-}}\NormalTok{ fingR}\SpecialCharTok{::}\FunctionTok{JAGS.pred}\NormalTok{(}\AttributeTok{path =} \FunctionTok{paste0}\NormalTok{(dir.mod.MixSIAR, }\StringTok{"contrib.csv"}\NormalTok{), }\CommentTok{\# location of files generated by \textasciigrave{}JAGS.summary\textasciigrave{}}
                                     \AttributeTok{stats =} \StringTok{"Median"}\NormalTok{,                               }\CommentTok{\# Summary statistics to calculate (Median or Mean)}
                                     \AttributeTok{save =} \ConstantTok{TRUE}\NormalTok{,                                    }\CommentTok{\# If the result should be saved}
                                     \CommentTok{\#note = "example"                               \# Optional: Additional note to append to the file name }
\NormalTok{                                     )}

\DocumentationTok{\#\# Ensure that the total predicted contribution sums to 1 or 100\%}
\NormalTok{MixSIAR.preds.mixE }\OtherTok{\textless{}{-}}\NormalTok{ fingR}\SpecialCharTok{::}\FunctionTok{ensure.total}\NormalTok{(}\AttributeTok{data =}\NormalTok{ MixSIAR.preds.mix,                  }\CommentTok{\# Predicted source contribution for each sample, data from fingR::BMM.pre}
                                          \AttributeTok{sample.name =} \StringTok{"sample"}\NormalTok{,                    }\CommentTok{\# Column name for sample identifier}
                                          \AttributeTok{path =}\NormalTok{ dir.mod.MixSIAR,                    }\CommentTok{\# Optional: Directory path for saving the results}
                                          \CommentTok{\#note = "example"                          \# Optional: Additional note to append to the file name}
\NormalTok{                                          )}
\end{Highlighting}
\end{Shaded}

\begin{Shaded}
\begin{Highlighting}[]
\NormalTok{MixSIAR.preds.mixE[}\DecValTok{1}\SpecialCharTok{:}\DecValTok{5}\NormalTok{,]}
\CommentTok{\#\textgreater{}              sample Median\_Forest Median\_Subsoil Median\_Undecontaminated total}
\CommentTok{\#\textgreater{} 1 ManoDd\_2106\_00{-}01         0.180          0.026                   0.794 1.000}
\CommentTok{\#\textgreater{} 2 ManoDd\_2106\_01{-}02         0.068          0.030                   0.902 1.000}
\CommentTok{\#\textgreater{} 3 ManoDd\_2106\_02{-}03         0.070          0.038                   0.892 0.998}
\CommentTok{\#\textgreater{} 4 ManoDd\_2106\_03{-}04         0.072          0.035                   0.893 0.999}
\CommentTok{\#\textgreater{} 5 ManoDd\_2106\_04{-}05         0.068          0.027                   0.905 0.999}
\end{Highlighting}
\end{Shaded}

Same code for virtual mixtures:

\begin{Shaded}
\begin{Highlighting}[]
\DocumentationTok{\#\# Summarise MixSIAR model previsions}
\NormalTok{MixSIAR.summary.VM }\OtherTok{\textless{}{-}}\NormalTok{ fingR}\SpecialCharTok{::}\FunctionTok{JAGS.summary}\NormalTok{(}\AttributeTok{jags.1 =}\NormalTok{ jags.VM,                            }\CommentTok{\# Data from the MixSIAR model \textasciigrave{}MixSIAR::run\_model()\textasciigrave{}}
                                          \AttributeTok{mix =}\NormalTok{ MSIAR.VM,                              }\CommentTok{\# Virtual mixtures dataset }
                                          \AttributeTok{sources =}\NormalTok{ MSIAR.source.VM,                   }\CommentTok{\# Source dataset loaded with virtual mixture mix}
                                          \AttributeTok{path =}\NormalTok{ dir.mod.MixSIAR,                      }\CommentTok{\# Directory path for saving the files}
                                          \AttributeTok{note =} \StringTok{"VM"}\NormalTok{,                                 }\CommentTok{\# Optional: Additional note to append to the file name}
                                          \AttributeTok{save\_pred =} \ConstantTok{TRUE}                             \CommentTok{\# Optional: Save the MixSIAR modelling predictions (heavy files)}
\NormalTok{                                          )}

\DocumentationTok{\#\# Extracts the median value of the previsions      }
\NormalTok{MixSIAR.preds.VM }\OtherTok{\textless{}{-}}\NormalTok{ fingR}\SpecialCharTok{::}\FunctionTok{JAGS.pred}\NormalTok{(}\AttributeTok{path =} \FunctionTok{paste0}\NormalTok{(dir.mod.MixSIAR, }\StringTok{"contrib\_VM.csv"}\NormalTok{), }\CommentTok{\# location of files generated by \textasciigrave{}JAGS.summary\textasciigrave{}}
                                     \AttributeTok{stats =} \StringTok{"Median"}\NormalTok{,                                 }\CommentTok{\# Summary statistics to calculate (Median or Mean)}
                                     \AttributeTok{save =} \ConstantTok{TRUE}\NormalTok{,                                      }\CommentTok{\# If the result should be saved}
                                     \AttributeTok{note =} \StringTok{"VM"}                                       \CommentTok{\# Optional: Additional note to append to the file name }
\NormalTok{                                     )}

\DocumentationTok{\#\# Ensure that the total predicted contribution sums to 1 or 100\%}
\NormalTok{MixSIAR.preds.VME }\OtherTok{\textless{}{-}}\NormalTok{ fingR}\SpecialCharTok{::}\FunctionTok{ensure.total}\NormalTok{(}\AttributeTok{data =}\NormalTok{ MixSIAR.preds.VM,                      }\CommentTok{\# Predicted source contribution for each sample, data from fingR::BMM.pre}
                                          \AttributeTok{sample.name =} \StringTok{"sample"}\NormalTok{,                      }\CommentTok{\# Column name for sample identifier}
                                          \AttributeTok{path =}\NormalTok{ dir.mod.MixSIAR,                      }\CommentTok{\# Optional: Directory path for saving the results}
                                          \AttributeTok{note =} \StringTok{"VM"}                                  \CommentTok{\# Optional: Additional note to append to the file name}
\NormalTok{                                          )}
\end{Highlighting}
\end{Shaded}

\begin{Shaded}
\begin{Highlighting}[]
\NormalTok{MixSIAR.preds.VME[}\DecValTok{1}\SpecialCharTok{:}\DecValTok{5}\NormalTok{,]}
\CommentTok{\#\textgreater{}   sample Median\_Forest Median\_Subsoil Median\_Undecontaminated total}
\CommentTok{\#\textgreater{} 1 VM{-}001         0.224          0.452                   0.324     1}
\CommentTok{\#\textgreater{} 2 VM{-}002         0.163          0.404                   0.433     1}
\CommentTok{\#\textgreater{} 3 VM{-}003         0.157          0.438                   0.405     1}
\CommentTok{\#\textgreater{} 4 VM{-}004         0.148          0.468                   0.384     1}
\CommentTok{\#\textgreater{} 5 VM{-}005         0.140          0.516                   0.344     1}
\end{Highlighting}
\end{Shaded}

\hypertarget{modelling-accuracy-statistics-1}{%
\paragraph{Modelling accuracy
statistics}\label{modelling-accuracy-statistics-1}}

The modelling accuracy of MixSIAR model is evaluate with the virtual
mixtures. These virtual mixtures, serving as target samples with known
contributions (\emph{VM.contrib}), allow for the calculation of
modelling accuracy metrics based on their prediction.

The \texttt{eval.groups} function calculates several common modelling
accuracy metrics: ME, RMSE, squared Pearson's correlation coefficient
(r2), and Nash-Sutcliff Modelling Efficiency Coefficient (NSE).

\begin{Shaded}
\begin{Highlighting}[]
\NormalTok{MixSIAR.stats }\OtherTok{\textless{}{-}}\NormalTok{ fingR}\SpecialCharTok{::}\FunctionTok{eval.groups}\NormalTok{(}\AttributeTok{df.obs =}\NormalTok{ VM.contrib,                                   }\CommentTok{\# Theoretical contribution}
                                    \AttributeTok{df.pred =}\NormalTok{ MixSIAR.preds.VME }\SpecialCharTok{\%\textgreater{}\%}\NormalTok{ dplyr}\SpecialCharTok{::}\FunctionTok{select}\NormalTok{(}\SpecialCharTok{{-}}\NormalTok{total), }\CommentTok{\# Predicted contribution (remove the $total column from ensured data.frame)}
                                    \AttributeTok{by =} \FunctionTok{c}\NormalTok{(}\StringTok{"Sample\_name"} \OtherTok{=} \StringTok{"sample"}\NormalTok{),                      }\CommentTok{\# Column where mixtures labels are specified (for \textasciigrave{}dplyr::left\_join\textasciigrave{} function)}
                                    \AttributeTok{path =}\NormalTok{ dir.mod.MixSIAR,                                }\CommentTok{\# Optional: Directory path for saving the results}
                                    \CommentTok{\#note = "example"                                      \# Optional: Additional note to append to the file name}
\NormalTok{                                    )}
\end{Highlighting}
\end{Shaded}

\begin{Shaded}
\begin{Highlighting}[]
\NormalTok{MixSIAR.stats}
\CommentTok{\#\textgreater{}     Type           Source    ME RMSE   r2  NSE}
\CommentTok{\#\textgreater{} 1 Median           Forest {-}0.09 0.16 0.83 0.62}
\CommentTok{\#\textgreater{} 2 Median          Subsoil  0.10 0.15 0.81 0.65}
\CommentTok{\#\textgreater{} 3 Median Undecontaminated {-}0.01 0.23 0.20 0.20}
\end{Highlighting}
\end{Shaded}

The \texttt{CRPS} functions calculate the continuous ranking probability
score and returns a list contraining two \emph{data.frame} objects; one
with the \texttt{\$samples} CRPS values per source class group (saved as
\emph{CRPS.csv}), the other is \texttt{\$mean} with the mean of the CRPS
per source class groups (saved as \emph{CRPS\_mean.csv}).

\begin{Shaded}
\begin{Highlighting}[]
\CommentTok{\# Calculate prediction CRPS values}
\NormalTok{MixSIAR.CRPS }\OtherTok{\textless{}{-}}\NormalTok{ fingR}\SpecialCharTok{::}\FunctionTok{CRPS}\NormalTok{(}\AttributeTok{obs =}\NormalTok{ VM.contrib,                                                     }\CommentTok{\# Observed contributions}
                            \AttributeTok{prev =} \FunctionTok{read.csv}\NormalTok{(}\FunctionTok{paste0}\NormalTok{(dir.mod.MixSIAR, }\StringTok{"MixSIAR\_prevision\_VM.csv"}\NormalTok{)), }\CommentTok{\# Predicted prevision from MixSIAR saved by \textasciigrave{}JAGS.summary()\textasciigrave{}}
                            \AttributeTok{source.groups =} \FunctionTok{c}\NormalTok{(}\StringTok{"Forest"}\NormalTok{, }\StringTok{"Subsoil"}\NormalTok{, }\StringTok{"Undecontaminated"}\NormalTok{),           }\CommentTok{\# Source class groups}
                            \AttributeTok{mean.cal =} \ConstantTok{TRUE}\NormalTok{,                                                      }\CommentTok{\# Calculate mean CRPS per source class group}
                            \AttributeTok{save.dir =}\NormalTok{ dir.mod.MixSIAR,                                           }\CommentTok{\# Optional: Directory path for saving the results}
                            \CommentTok{\#note = "example"                                                     \# Optional: Additional note to append to the file name}
\NormalTok{                            )}
\end{Highlighting}
\end{Shaded}

\begin{Shaded}
\begin{Highlighting}[]
\NormalTok{MixSIAR.CRPS}\SpecialCharTok{$}\NormalTok{samples[}\DecValTok{1}\SpecialCharTok{:}\DecValTok{6}\NormalTok{,]}
\CommentTok{\#\textgreater{}   Sample\_name Forest Subsoil Undecontaminated}
\CommentTok{\#\textgreater{} 1      VM{-}001 0.2136  0.4360           0.6565}
\CommentTok{\#\textgreater{} 2      VM{-}002 0.1142  0.2757           0.3977}
\CommentTok{\#\textgreater{} 3      VM{-}003 0.1150  0.2650           0.3870}
\CommentTok{\#\textgreater{} 4      VM{-}004 0.1063  0.2487           0.3627}
\CommentTok{\#\textgreater{} 5      VM{-}005 0.1031  0.2467           0.3568}
\CommentTok{\#\textgreater{} 6      VM{-}006 0.0966  0.2354           0.3393}

\NormalTok{MixSIAR.CRPS}\SpecialCharTok{$}\NormalTok{mean}
\CommentTok{\#\textgreater{}             Source CRPS.mean}
\CommentTok{\#\textgreater{} 1           Forest    0.0959}
\CommentTok{\#\textgreater{} 2          Subsoil    0.0862}
\CommentTok{\#\textgreater{} 3 Undecontaminated    0.1310}
\end{Highlighting}
\end{Shaded}

The \texttt{interval.width} functions calculate two prediction interval
width: The \emph{W50} contains 50\% of the prevision (Q75-Q25) and the
\emph{W95} contains 95\% of the prevision (Q97.5-Q2.5). It returns a
list contraining two \emph{data.frame} objects; one with the
\texttt{\$samples} prediction interval width values per source class
group (saved as \emph{Interval\_width.csv}), the other is
\texttt{\$mean} with the mean of the prediction interval width per
source class groups (saved as \emph{Interval\_width\_mean.csv}).

\begin{Shaded}
\begin{Highlighting}[]
\CommentTok{\# Calculate prediction interval width (W95, W50)}
\NormalTok{MixSIAR.predWidth }\OtherTok{\textless{}{-}}\NormalTok{ fingR}\SpecialCharTok{::}\FunctionTok{interval.width}\NormalTok{(}\AttributeTok{path.to.prev =} \FunctionTok{paste0}\NormalTok{(dir.mod.MixSIAR, }\StringTok{"MixSIAR\_prevision\_VM.csv"}\NormalTok{), }\CommentTok{\# Predicted prevision from MixSIAR saved by \textasciigrave{}JAGS.summary()\textasciigrave{}}
                                           \AttributeTok{mean.cal =} \ConstantTok{TRUE}\NormalTok{,                                                    }\CommentTok{\# Calculate mean of interval width per source group}
                                           \AttributeTok{save =} \ConstantTok{TRUE}\NormalTok{,                                                        }\CommentTok{\# Save the results at the same location of the path.to.prev}
                                           \CommentTok{\#note = "exemple"                                                   \# Optional: Additional note to append to the file name}
\NormalTok{                                           )}
\end{Highlighting}
\end{Shaded}

\begin{Shaded}
\begin{Highlighting}[]
\NormalTok{BMM.predWidth}\SpecialCharTok{$}\NormalTok{samples[}\DecValTok{1}\SpecialCharTok{:}\DecValTok{6}\NormalTok{,]}
\CommentTok{\#\textgreater{}   mix.names           source   W50   W95}
\CommentTok{\#\textgreater{} 1    VM{-}001           Forest 0.154 0.796}
\CommentTok{\#\textgreater{} 2    VM{-}001          Subsoil 0.728 0.997}
\CommentTok{\#\textgreater{} 3    VM{-}001 Undecontaminated 0.614 0.990}
\CommentTok{\#\textgreater{} 4    VM{-}002           Forest 0.297 0.768}
\CommentTok{\#\textgreater{} 5    VM{-}002          Subsoil 0.623 0.997}
\CommentTok{\#\textgreater{} 6    VM{-}002 Undecontaminated 0.652 0.942}

\NormalTok{BMM.predWidth}\SpecialCharTok{$}\NormalTok{mean}
\CommentTok{\#\textgreater{} \# A tibble: 3 x 3}
\CommentTok{\#\textgreater{}   Source           W50.mean W95.mean}
\CommentTok{\#\textgreater{}   \textless{}chr\textgreater{}               \textless{}dbl\textgreater{}    \textless{}dbl\textgreater{}}
\CommentTok{\#\textgreater{} 1 Forest              0.356    0.883}
\CommentTok{\#\textgreater{} 2 Subsoil             0.59     0.969}
\CommentTok{\#\textgreater{} 3 Undecontaminated    0.408    0.914}
\end{Highlighting}
\end{Shaded}

The \texttt{ESP} function calculates the Encompassed Sample Prediction
(ESP). The ESP is a newly introduced statistics in
\href{}{Chalaux-Clergue et al (under review)} and was created to assess
the transferability of the statistics calculated on virtual mixtures to
actual sediment samples. The ESP was calculated as the percentage of
actual samples for which the predicted contributions remained within the
lowest and the highest predicted contributions obtained for the virtual
mixtures. When expressed as a percentage, ESP ranges from 0 to 100\%,
the latter providing an optimal value. Values close to 100\% indicate a
higher transferability of modelling evaluation statistics calculated on
virtual mixture to actual sediment samples.

\begin{Shaded}
\begin{Highlighting}[]
\NormalTok{sources.lvl }\OtherTok{\textless{}{-}} \FunctionTok{c}\NormalTok{(}\StringTok{"Forest"}\NormalTok{, }\StringTok{"Subsoil"}\NormalTok{, }\StringTok{"Undecontaminated"}\NormalTok{)}

\CommentTok{\# Calculate encompassed sample predictions (ESP)}
\NormalTok{MixSIAR.ESP }\OtherTok{\textless{}{-}}\NormalTok{ fingR}\SpecialCharTok{::}\FunctionTok{ESP}\NormalTok{(}\AttributeTok{obs =}\NormalTok{ MixSIAR.preds.VM,                   }\CommentTok{\# Virtual mixtures predicted contributions}
                          \AttributeTok{pred =}\NormalTok{ MixSIAR.preds.mixE,                }\CommentTok{\# Actual sediment samples predicted contributions}
                          \AttributeTok{sources =} \FunctionTok{paste0}\NormalTok{(}\StringTok{"Median\_"}\NormalTok{, sources.lvl), }\CommentTok{\# Sources labels in prediction objects}
                          \AttributeTok{count =} \StringTok{"Both"}                            \CommentTok{\# Count \textquotesingle{}Number\textquotesingle{} and \textquotesingle{}Percentage\textquotesingle{}}
\NormalTok{                          )}
\end{Highlighting}
\end{Shaded}

\begin{Shaded}
\begin{Highlighting}[]
\NormalTok{MixSIAR.ESP}
\CommentTok{\#\textgreater{}                                   Source ESP.Number ESP.Percentage}
\CommentTok{\#\textgreater{} Median\_Forest                     Forest         38            100}
\CommentTok{\#\textgreater{} Median\_Subsoil                   Subsoil          0              0}
\CommentTok{\#\textgreater{} Median\_Undecontaminated Undecontaminated          4             11}
\end{Highlighting}
\end{Shaded}

Modelling accuracy statistics could be interpreted the following way:
``Higher values of W50 indicate a wider distribution, which is related
to a higher uncertainty. The sign of the ME indicates the direction of
the bias, i.e.~an overestimation or underestimation (positive or
negative value, respectively). As ME is affected by cancellation, a ME
of zero can also reflect a balanced distribution of predictions around
the 1 : 1 line. Although this is not a bias, it does not mean that the
model outputs are devoid of errors. The RMSE is a measure of the
accuracy and allows us to calculate prediction errors of different
models for a particular dataset. RMSE is always positive, and its ideal
value is zero, which indicates a perfect fit to the data. As RMSE
depends on the squared error, it is sensitive to outliers. The r2
describes how linear the prediction is. The NSE indicates the magnitude
of variance explained by the model, i.e.~how well the predictions match
with the observations. A negative RMSE indicates that the mean of the
measured values provides a better predictor than the model. The joint
use of r2 and NSE allows for a better appreciation of the distribution
shape of predictions and thus facilitates the understanding of the
nature of model prediction errors. The CRPS evaluates both the accuracy
and sharpness (i.e.~precision) of a distribution of predicted continuous
values from a probabilistic model for each sample (Matheson and Winkler,
1976). The CRPS is minimised when the observed value corresponds to a
high probability value in the distribution of model outputs.''
\href{10.5194/soil-10-109-2024}{(Chalaux-Clergue et al, 2024)}.

\hypertarget{future-updates}{%
\subsection{Future updates}\label{future-updates}}

Upcoming updates will introduce graphical support functions such as
\emph{Bayesian prediction density plots}, \emph{prediction
vs.~observation plots}, and \emph{ternary diagrams}.

\hypertarget{getting-help}{%
\subsection{Getting help}\label{getting-help}}

If you encounter a clear bug, please file and issue or send an email to
\href{mailto:thomaschalaux@icloud.com,\%20rbizeul59@gmail.com}{Thomas
Chalaux-Clergue and Rémi Bizeul}.

\hypertarget{citation}{%
\subsection{Citation}\label{citation}}

To cite this packages:

\begin{Shaded}
\begin{Highlighting}[]
\NormalTok{utils}\SpecialCharTok{::}\FunctionTok{citation}\NormalTok{(}\AttributeTok{package =} \StringTok{"fingR"}\NormalTok{)}
\CommentTok{\#\textgreater{} To cite the \textquotesingle{}fingR\textquotesingle{} package in publications please use:}
\CommentTok{\#\textgreater{} }
\CommentTok{\#\textgreater{}   Chalaux{-}Clergue, T. and Bizeul, R (2024). fingR: A package to support}
\CommentTok{\#\textgreater{}   sediment source fingerprinting studies, Zenodo [Package]:}
\CommentTok{\#\textgreater{}   https://doi.org/10.5281/zenodo.8293595, Github [Package]:}
\CommentTok{\#\textgreater{}   https://github.com/tchalauxclergue/fingR, Version = 2.0.0.}
\CommentTok{\#\textgreater{} }
\CommentTok{\#\textgreater{} Une entrée BibTeX pour les utilisateurs LaTeX est}
\CommentTok{\#\textgreater{} }
\CommentTok{\#\textgreater{}   @Manual\{,}
\CommentTok{\#\textgreater{}     title = \{fingR: A package to support sediment source fingerprinting studies\},}
\CommentTok{\#\textgreater{}     author = \{\{Chalaux{-}Clergue\} and \{Thomas\} and \{Bizeul\} and \{Rémi\}\},}
\CommentTok{\#\textgreater{}     year = \{2024\},}
\CommentTok{\#\textgreater{}     month = \{6\},}
\CommentTok{\#\textgreater{}     note = \{R package version 2.0.0\},}
\CommentTok{\#\textgreater{}     doi = \{https://doi.org/10.5281/zenodo.8293595\},}
\CommentTok{\#\textgreater{}     url = \{https://github.com/tchalauxclergue/fingR\},}
\CommentTok{\#\textgreater{}   \}}
\end{Highlighting}
\end{Shaded}

\hypertarget{references}{%
\subsection{References}\label{references}}

\begin{itemize}
\item
  Chalaux-Clergue, T., Bizeul, R., Foucher, A., \& Evrard, O. (2024a).
  An unified template for sediment source fingerprinting databases
  (24.03.01) {[}Data set{]}. Zenodo.
  \url{https://doi.org/10.5281/zenodo.10725787}.
\item
  Chalaux-Clergue, T., Evrard, O., Durand, R., Caumon, A., Hayashi, S.,
  Tsuji, H., Huon, S., Vaury, V., Wakiyama, Y., Nakao, A., Laceby, J.
  P., \& Onda, Y. (2024b). Organic matter, geochemical, visible
  spectrocolorimetric properties, radiocesium properties, and grain size
  of potential source material, target sediment core layers and
  laboratory mixtures for conducting sediment fingerprinting approaches
  in the Mano Dam Reservoir (Hayama Lake) catchment, Fukushima
  Prefecture, Japan (Version 2) {[}Data set{]}. Zenodo.
  \url{https://doi.org/10.5281/zenodo.10836974}.
\item
  Chalaux-Clergue, T., Bizeul, R., Batista, P. V. G., Martinez-Carreras,
  N., Laceby, J. P., Evrard, P. (2024c). Sensitivity of source sediment
  fingerprinting to tracer selection. SOIL, 10(1), 109-138.
  \url{https://doi.org/10.5194/soil-10-109-2024}.
\item
  Chalaux-Clergue, T., \& Bizeul, R. (2024d). fingR: A support for
  sediment source fingerprinting studies (All version). Zenodo.
  \url{https://doi.org/10.5281/zenodo.8293595}. Github.
  \url{https://github.com/tchalauxclergue/fingR}.
\item
  Laceby JP, Olley J, Pietsch TJ, Sheldon F, Bunn SE. Identifying
  subsoil sediment sources with carbon and nitrogen stable isotope
  ratios. Hydrological Processes. 15 avr 2015;29(8):1956‑71.
  \url{https://doi.org/10.1002/hyp.10311}
\item
  Stock, B. C., Jackson, A. L., Ward, E. J., Parnell, A. C., Phillips,
  D. L., \& Semmens, B. X. (2018). Analyzing mixing systems using a new
  generation of Bayesian tracer mixing models. PeerJ, 6, e5096.
  \url{https://doi.org/10.7717/peerj.5096}.
\item
  Stock, B. C., Jackson, A. L., Ward, E. J., Parnell, A. C., Phillips,
  D. L. (2020). MixSIAR: Bayesian Mixing Models in R (Version 3.1.12).
  Zenodo. \url{https://doi.org/10.5281/zenodo.594910}. Github.
  \url{https://github.com/brianstock/MixSIAR/tree/3.1.11}
\end{itemize}

\end{document}
